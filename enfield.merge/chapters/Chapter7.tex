\chapter{Conclusion}



Ever since Darwin's\is{Darwin} earliest remarks on the uncanny similarity between 
language change\is{language change} and natural history in biology\is{evolution!biological}, there has been a 
persistent conceptual unclarity in evolutionary approaches to cultural 
change. This unclarity concerns the units of analysis. 



In some cases the unit is said to be the language system as a whole.\is{languages, as unit of analysis} A 
language, then, is ``like a species'' (\citealt[60]{darwin_descent_1871}; cf. \citealt[192--194]{mufwene_ecology_2001}). If so, then we are talking about a population 
of idiolects\is{idiolects} that is coterminous with a population of bodies (allowing, 
of course, that in the typical situation -- multilingualism\is{multilingualism} -- one body houses 
more than one linguistic system). 



On another view, the unit of analysis is any unit that forms \textit{part} of a language, such as a word or a piece of grammar.\is{linguistic items} \textquoteleft A struggle 
for life is constantly going on amongst the words and grammatical forms 
in each language' (\citealt{muller_darwinism_1870}, cited in \citealt[60]{darwin_descent_1871}). By
contrast with the idea of populations of idiolects, this suggests that 
there are populations \textit{of items }(akin to Zipf's economy of 
word-tools), where these items are produced, and perceived, in the 
context of spoken utterances. 



While some of us instinctively think first in terms of items,\is{linguistic items} and others 
of us first in terms of systems,\is{systems} we do not have the luxury of ignoring either. 
Neither an item nor a system can exist without the other. The challenge 
is to characterize the relation between the two. This relation is the 
one thing that defines them both. 



The issue is not just the relative status of items and systems but the 
causal relations between them. If the distinction 
between item and system is a matter of framing, it is no less 
consequential for that. We not only have to define the differences 
between item phenomena and system phenomena, we must know which ones we 
are talking about and when. And we must show whether, and if so how, we 
can translate statements about one into statements about the other. 

\section{Natural causes of language}


\textquoteleft We might gain considerable insight into the mainsprings of human 
behavior', wrote \citet[v]{zipf_human_1949}, \textquoteleft if we viewed it purely as a natural 
phenomenon like everything else in the universe'. This does not mean 
that we cannot embrace the anthropocentrism, subjectivity, and 
self-reflexivity of human affairs. It does mean that underneath all of 
that, our analyses remain accountable to natural, causal claims. In this book we have developed a causally explicit model for the 
transmission of cultural items, and we have approached a solution to the 
item/system problem\is{item/system problem} that builds solely on these item-based biases. I 
submit that the biases\is{transmission biases} required for item evolution -- never forgetting that 
``item'' here really means 
``something-and-its-functional-relation-to-a-context'' -- are sufficient not 
only to account for how and why certain cultural items win or lose. They 
also account for the key relational forces that link items 
with systems. 

%If we are going to answer the two fundamental questions \textquoteleft What's language like?' and \textquoteleft Why is it like that?' we will have to look in multiple places for answers. Our case study of the length-frequency\is{length-frequency rule} correlation in words (section \ref{zipflengthrule}) invoked cognitive preferences\is{cognitive biases} of individual people and related these to formal features of community-wide systems. To establish links between these, we drew on a set of frames -- under the rubric of MOPEDS\is{MOPEDS framework} -- that are distinguished from each other in terms of types of causal process and conditional structure, and that roughly correlate with different timescales\is{time scales} (Chapter \ref{causaldynamics}). These different types of process together account for how meaning arises in the moment.

We have confronted the item/system problem. To solve it, we 
reached for the most tangible known causal mechanism for the existence 
of linguistic and cultural reality: item-based transmission. The outcome is this. With the right definition of `item' -- as always having a functional relation to context -- we can have 
an item-based account for linguistic and cultural reality that gives us 
a system ontology for free.\is{ontology of language}




\section{Toward a framework}

Why do neighboring languages share structures in common? In earlier work on language contact,\is{language contact} maintenance, and change\is{language change} \citep{enfield_linguistic_2003,enfield_areal_2005,enfield_transmission_2008,enfield_linguistic_2011}, I considered some of the challenges that this question raises. This led me to confront the conceptual problems I have discussed in this book. They are problems of causality. What makes languages the way they are? What causes a language to have certain features and not others? How permeable are language systems?\is{systems} These questions led me to look for a causal account of the ontology of language.\is{ontology of language} I have tried in the above chapters to present some of the ideas that came out. Together, these ideas suggest a natural, causal framework for understanding the foundations of language. The framework has two conceptual components:
%\footnote{On innovation diffusion, see Rogers (2003); on language and social networks, see Milroy (1980), Le Page and Tabouret-Keller (1985), Ross (1997), and Nettle (1999); on cultural evolution, see Sperber (2006), Boyd and Richerson (2005)}
\begin{enumerate}
\item[] {\textit{Causal frames}:\is{causal frames} There are multiple frames or ``time-scales'' within which change in linguistic and other cultural systems can be causally affected. While most approaches work within just one or two of these frames, all of these frames should be considered together, with special attention to the links between them. As explicated in Chapter \ref{causaldynamics}, the framework recognizes six such frames, under the rubric of MOPEDS\is{MOPEDS framework}: \textit{microgenetic},\is{microgenetic frame} invoking cognitive and motoric processes for producing and comprehending language and other goal-directed behavior; \textit{ontogenetic},\is{ontogenetic frame} invoking lifespan processes by which people, usually as children, acquire linguistic and cultural knowledge and skills; \textit{phylogenetic},\is{phylogenetic frame} invoking ways in which the requisite cognitive capacities have evolved in our species;\is{evolution!biological} \textit{enchronic},\is{enchronic frame} invoking the sequential interlocking of social actions in linguistic clothing; \textit{diachronic},\is{diachronic frame} invoking historical change, conducted socially in human populations; and \textit{synchronic},\is{synchronic frame} any approach, such as linguistic or ethnographic description, that does not explicitly invoke notions of process.}

\item[] {\textit{Transmission biases}:\is{transmission biases} A socially- and cognitively-grounded account of the genesis, diffusion, and conventionalization of innovations\is{innovation} in human populations must provide a causal basis for how it is that social conventions -- such as the linguistic and ethnographic facts that we observe -- are the way they are. As explicated in Chapter \ref{Transmission biases}, the causal machinery for diffusion\is{diffusion} of types of behavior (including language) within a population is a driving force -- an engine of sorts -- with four linked loci: \textit{exposure}\is{exposure} to a bit of behavior, \textit{representation}\is{representation} of that bit of behavior, subsequent \textit{reproduction}\is{reproduction} of that bit of behavior, and \textit{material}\is{material} instantiation of some trace of the behavior (leading to exposure of others, feeding back into the process anew). Each locus is a site where the chain of diffusion may be broken, reinforced, or transformed: Such breaks, reinforcements, and transformations come from \textit{biases}\is{transmission biases} that may operate on each locus (Chapter \ref{Transmission biases} gives the details). There are many of these biases. Some are cognitive. For example, if a linguistic construction is easier to learn, it will diffuse better. Some are social. For example, if more prestigious people model an innovation, other people are more likely to copy it. }
\end{enumerate}

These two conceptual pillars of a framework for understanding the natural causes of language should be enough to provide the raw materials for explaining the ontology of linguistic systems.\is{ontology of language} 

Linguistic system ontology is a puzzle because items (in contexts) are the only things that circulate and yet somehow systems exist. If our conceptual framework recognizes multiple coexisting causal frames and multiple coexisting loci of transmission, it becomes possible to see how gaps and interfaces between these frames and loci provide the traction for system emergence. At least it becomes possible to study the problem. Empirical and theoretical investigations will have to draw not only on the linguistics of descriptive grammar, semantics, pragmatics, and typology, but also on sociological research on innovation diffusion,\is{diffusion} sociolinguistic\is{sociolinguistics} research on social networks,\is{social networks} and the natural science of cultural evolution. A framework like this should allow us to be maximally explicit about the causal processes that create linguistic and other cultural facts.


