\chapter{Causal frames}
\label{causaldynamics}



If you really want to understand language, you will have to study a lot of different things. Here are some:

\begin{enumerate}
\item[\textbullet] {The finely-timed perceptual, cognitive, and motoric processes involved in producing and comprehending language}\is{processing}
\item[\textbullet] {The early lifespan processes by which children learn linguistic and communicative knowledge and skills}\is{first language acquisition}
\item[\textbullet] {The evolutionary\is{evolution!biological} processes that led to the unique emergence of the cognitive capacities for language in our species}
\item[\textbullet] {The ways in which the things we say are moves in sequences of social actions}\is{social action}
\item[\textbullet] {The mechanisms and products of language change, with links between historical processes and evolutionary processes}\is{language change}\is{evolution!historical}
\item[\textbullet] {Linguistic variation\is{variation} and its role in how historical change in language takes place in human populations}
\item[\textbullet] {Things that can be described without reference to process or causation at all, as seen in linguistic grammars,\is{grammars} dictionaries\is{dictionaries}, ethnographies\is{ethnographies}, and typologies\is{typologies}, where relationships rather than processes are the focus}
\end{enumerate}
%\footnote{On language production and comprehension, see for example Levelt 1989, 2012; Emmorey 2002; McNeill 2005; Cutler 2012. On language acquisition see Brown and  Gaskins, 2014; Schieffelin and Ochs 1986; Tomasello 2003. On language evolution see Hurford 2007, 2012;  Tomasello 2008; Hauser, Chomsky, and Fitch 2002; Chomsky 2011. On conversational sequence, see Schegloff 1968; Goffman  1981a; Goodwin 2000, 2006; Sidnell and Stivers 2012; Enfield 2013;  Enfield and Sidnell 2014; Sidnell and Enfield 2014. On language change, see Dixon  1997; Harris and Campbell 1995; Hopper and Traugott 1993; Hanks 2010; for evolutionary perspectives, see Boyd and Richerson 1985, 2005; Durham 1991; Smith, Brighton, and Kirby 2003; for the role of variation in change, see Labov 2011;  Trudgill 2010; Eckert 2000.}

These different points of focus correspond roughly with distinct research 
perspectives. But they do not merely represent disciplinary 
alternatives. The different perspectives can be seen to 
fit together as parts of a larger conceptual framework. 



To give some outline to that framework, I here define six interconnected frames for orienting our work. They remind us of the 
perspectives that are always available and potentially relevant, but that we might 
not be focusing on. They do not constitute a definitive set of 
frames -- there is no definitive set -- but they are useful. They correspond 
well to the most important causal domains.\is{causal frames} They conveniently group 
similar or tightly interconnected sets of causal mechanism under single 
rubrics. And together they cover most of what we need for providing 
answers to our questions in research on language. 


The frames are \textit{Microgenetic}, \textit{Ontogenetic}, 
\textit{Phylogenetic}, \textit{Enchronic}, \textit{Diachronic}, 
and \textit{Synchronic}. The meanings of these terms are explicated below. As a mnemonic, they spell MOPEDS\is{MOPEDS framework}. Frames like these are sometimes referred to as \textit{time scales}\is{time scales}. But calling them `scales' is not accurate. It implies that they all measure the same thing, just with arbitrarily different units of measure -- seconds versus minutes versus hours, etc. But the difference between, say, ontogenetic and diachronic (ditto for the other frames) is not defined in terms of abstract or objective units of the same underlying stuff -- time, in this case. The frames are defined and distinguished in 
terms of different types of underlying processes and 
causal-conditional mechanisms. For each frame, what 
matters most is how it works, not how long it takes. 



By offering a scheme of interrelated causal frames as part of a conceptual 
framework for research on language, I want to stress two points. 



The first is that these frames are most useful when we keep them 
conceptually distinct. Kinds of reasoning that apply within one frame do 
not necessarily apply in another, and data that are relevant in one 
frame might not be relevant (in the same ways) in another. Mixing up these frames leads to confusion. 



The second point is that for a full understanding of the 
things we study it is not enough just to understand these 
things from within all of the different frames. The ideal is also to 
show how each frame is linked to each other frame, and, ultimately, how 
together the frames reveal a system of causal forces 
that define linguistic reality.



\section{Distinct frames and forces}
\label{Distinctframesandforces}

The ethologist Niko Tinbergen famously emphasized that different kinds of research question may be 
posed within different theoretical and methodological frames, and may 
draw on different kinds of data and reasoning \citep{tinbergen_aims_1963}. See \tabref{tinbergenfour}.



\begin{table}[h]
\centering
\begin{tabular}{ll}
\lsptoprule
\textbf{Causal} & What is the mechanism by which the behavior occurs?
\\
%\hline
\textbf{Functional} & What is the survival or fitness value of the 
behavior? \\
%\hline
\textbf{Phylogenetic} & How did the behavior emerge in the course of 
evolution? \\
%\hline
\textbf{Ontogenetic} & How does the behavior emerge in an individual's 
lifetime? \\
\lspbottomrule
\end{tabular}
\caption{Distinct causal/temporal frames for studying animal 
behavior, after \citet{tinbergen_aims_1963}.}
\label{tinbergenfour}
\end{table}



Tinbergen's four questions were applied in studying the behavior of 
non human animals. The distinctions were designed to handle 
communication systems such as the mating behavior of stickleback fish\is{stickleback fish}, 
not the far greater complexities of language, nor the rich cultural 
contexts of language systems. If we are going to capture the spirit of 
Tinbergen's idea, we need a scheme that better covers the phenomena 
specific to language and its relation to human diversity\is{diversity}. 


Many researchers of language and culture have emphasized the need to monitor and distinguish different causal frames that determine our perspective. These include researchers of last century \citep{saussure_cours_1916,vygotsky_thought_1962} through 
to many of today \citep{tomasello_constructing_2003,macwhinney_emergence_2005,raczaszek-leonardi_multiple_2010,cole_phylogeny_2007,donald_slow_2007,larsen-freeman_complexity_2008,uryu_ecology_2014,lemke_across_2000,lemke_language_2002}. We now consider some of the distinctions they have offered.



The classical two-way distinction made by \citet{saussure_cours_1916} -- \textit{synchronic}\is{synchronic frame} versus \textit{diachronic}\is{diachronic frame} -- is the tip of the iceberg. In a synchronic frame, we view language as a static 
system of relations. In a diachronic frame, we look at the historical processes of change that give rise to the synchronic relations observed. But if you look at the dynamic nature of language you will quickly see that 
diachrony -- in the usual sense of the development and 
divergence of languages through social history -- is not the only dynamic frame. 



Vygotsky distinguished between \textit{phylogenetic}\is{phylogenetic frame}, \textit{ontogenetic}\is{ontogenetic frame}, and \textit{historical} processes, and stressed that 
these dynamic frames were distinct from each other yet interconnected. 
His insight has been echoed and developed, from 
psychologists of communication like \citet{tomasello_cultural_1999} and \citet{cole_phylogeny_2007} to computational linguists like \citet{steels_synthesizing_1998,steels_evolving_2003} and \citet{smith_complex_2003}. 



\citet[540]{smith_complex_2003} argue that to understand language we have to see it 
as emerging out of the interaction of multiple complex adaptive systems. They name three ``time scales''\is{time scales} that need to be taken into account -- \textit{phylogenetic}\is{phylogenetic frame}, \textit{ontogenetic}\is{ontogenetic frame}, and \textit{glossogenetic}\is{diachronic frame} 
(= ``cultural evolution'', i.e., diachronic) -- thus echoing 
Vygotsky. Language is, they write, \textquoteleft a consequence of the interaction 
between biological evolution, learning and cultural evolution' \citep[541]{smith_complex_2003}. R\k{a}czaszek-Leonardi focuses on psycholinguistic\is{psycholinguistics} research, and proposes that three frames need to be addressed: \textit{online}, \textit{ontogenetic} and \textit{diachronic}. She leaves out the phylogenetic frame, but adds the ``online'' frame 
of cognitive processing\is{processing}. \citet[185]{cole_cultural_1996} expands the list of dynamic frames to include \textit{microgenesis}, \textit{ontogeny} (distinguishing early learning from 
overall lifespan), \textit{cultural history}, \textit{phylogeny}, 
and even \textit{geological time}. \citet[193--195]{macwhinney_emergence_2005} offers a list of \textquoteleft seven markedly different 
time frames for emergent processes and structure', citing Tinbergen's 
mentor Konrad \citet{lorenz_evolution_1958}. MacWhinney's frames are \textit{phylogenetic}, \textit{epigenetic}, \textit{developmental}, 
\textit{processing}, \textit{social}, \textit{interactional}, 
and \textit{diachronic}. 



\citet[122]{newell_unified_1990} proposes a somewhat more mechanical division of time 
into distinct ``bands of cognition'' (each consisting of three ``scales''). Newell takes the abstract/objective temporal unit of the second as a key 
unit, and defines each timescale\is{time scales} on a gradient from 10$^{-4}$ 
seconds at the fast end to 10$^{7}$ seconds at the slow end: the 
\textit{biological band} (= 10$^{-4}$-10$^{-2}$ seconds), the 
\textit{cognitive band} (= 10$^{-1}$-10$^{1}$ seconds), the 
\textit{rational band} (= 10$^{2}$-10$^{4}$ seconds), and the 
\textit{social band} (= 10$^{5}$-10$^{7}$ seconds). He also 
adds two ``speculative higher bands'': the \textit{historical band} (= 
10$^{8}$-10$^{10}$ seconds), and the \textit{evolutionary band} 
(= 10$^{11}$-10$^{13}$ seconds; \citealt[152]{newell_unified_1990}), thus suggesting a 
total of 18 distinct timescales. 



Like Newell (though without reference to him), \citet[277]{lemke_across_2000} takes 
the second as his unit and proposes no less than 24 \textquoteleft representative 
timescales',\is{time scales} beginning with 10$^{-5}$ seconds -- at which a typical 
process would be ``chemical synthesis'' -- through to 10$^{18}$ 
seconds -- the scale of ``cosmological processes''. 



Lemke's discussion is full of insights. But he generates his taxonomy by arbitrarily carving up an abstract gradient. It is not established in terms of research-relevant qualitative distinctions or 
methodological utility, nor is it derived from a theory (cf.  
Uryu et al \citeyear{uryu_ecology_2014}, \citeyear[169]{larsen-freeman_complexity_2008}). It is not clear, for example, why a distinction between units of 3.2 years 
versus 32 years should necessarily correlate with a distinction between 
processes like institutional planning versus identity change; nor why 
the process of evolutionary change should span three timescales (3.2 
million years, 32 million years and 317 million years) or why it should 
not apply at other timescales. 



\citet[169]{larsen-freeman_complexity_2008} propose a set of \textquoteleft timescales\is{time scales} 
relevant to face-to-face conversation between two people': a \textit{mental processing }timescale of milliseconds, a \textit{microgenetic }
timescale of online talk, a \textit{discourse event} timescale, a 
\textit{series of connected discourse events}, an \textit{ontogenetic }scale of an individual's life, and a \textit{phylogenetic} timescale. \citet{uryu_ecology_2014} critique this model for not explaining why these 
timescales are the salient or relevant ones, and for not specifying 
which other timescales are ``real but irrelevant''. 



\citet{uryu_ecology_2014} propose a principled ``continuum'' of timescales\is{time scales} running from 
``fast'' to ``slow'' (11 distinctions in the order \textit{atomic}, 
\textit{metabolic}, \textit{emotional}, \textit{autobiographical}, \textit{interbodily}, \textit{microsocial}, \textit{event}, 
\textit{social systems}, \textit{cultural}, \textit{evolutionary}, \textit{galactic}) that are orthogonal to a set of \textquoteleft temporal 
ranges' running from ``simple'' to ``complex'' (six distinctions in the 
order \textit{physical universe}, \textit{organic life forms}, 
\textit{human species}, \textit{human phenotype}, \textit{
dialogical system}, \textit{awareness}). Uryu et al's approach 
applies the notion of ecology to the dynamics of language and its usage (see also \citealt{cowley_distributed_2011}, \citealt{steffenson_ecolinguistics:_2013}).



What to make of this array of multi-scale schemes? Some are well-motivated but incomplete. Saussure gives 
a single dynamic frame, leading us to wonder, for example, whether we 
should regard speech processing as nano-diachrony\is{nano-diachrony}. Vygotsky gives us three dynamic frames, but does not single out or 
sub-distinguish ``faster'' frames like microgeny\is{microgenetic frame} and enchrony\is{enchronic frame}. Are we to 
think of these as pico-ontogeny?\is{pico-ontogeny} On the other hand, some schemes give us 
finer differentiation than we need, or offer arbitrary 
motivations for the distinctions made. What we need is a middle way. 



\section{MOPEDS: A basic-level set of causal frames}
\is{MOPEDS framework}

Of the frames discussed in the previous section, six capture 
what is most useful about previous proposals. These six frames are relatively well understood. They are known to be 
relevant to research. They are well-grounded in prior work on language and 
culture. And they are known to be related to each other in interesting ways.\footnote{One might wonder if one or more of these frames might be reduced in terms of one or more others. It is reminiscent of the idea of reducing social processes to physical ones: Were such a reduction possible, it is unlikely to be helpful.} This is what we need: a basic-level set of conceptually 
distinct but interconnected causal frames for understanding language. 



Each of the six frames -- microgenetic, ontogenetic, phylogenetic, 
enchronic, diachronic, synchronic -- is distinct from the others in terms 
of the kinds of causality it implies, and thus in its relevance to what 
we are asking about language and its relation to culture and other 
aspects of human diversity. One way to think about these distinct frames 
is that they are different sources of evidence for explaining the things 
that we want to understand. I now briefly define each of the six frames.



\subsection{Microgenetic (action processing)}
\is{microgenetic frame}


In a microgenetic frame, we look at how language and culture are psychologically processed. For example, in order to produce a simple sentence, a person goes through a set of 
cognitive processes including concept formulation, lemma retrieval, and phonological 
encoding \citep{levelt_speaking:_1989}. Or when we hear and understand what someone says \citep{cutler_native_2012}, we have to parse the 
speech stream, recognize distinct words and constructions, and infer 
others' communicative intentions. 

These processes tend to take place at time scales between a few 
milliseconds and a few seconds. Causal mechanisms at this 
level include working memory \citep{baddeley_working_1986}, rational 
heuristics \citep{gigerenzer_heuristics:_2011}, minimization of
effort \citep{zipf_human_1949}, categorization, motor routines, 
inference, ascription of mental states such as beliefs, 
desires, and intentions \citep{searle_intentionality:_1983,enfield_roots_2006}, and 
the fine timing of motor control and action execution.



\subsection{Ontogenetic (biography)}
\is{ontogenetic frame}


In an ontogenetic frame, we look at how a 
person's linguistic habits and abilities are learned and
developed during the course of that person's lifetime. Many of the 
things that are studied within this frame come under the general 
headings of language acquisition and socialization. This refers to both
the learning of a first language by infants\is{first language acquisition} (see \citealt{clark_first_2009}, \citealt{brown_language_2014}) and the learning of a second 
language by adults \citep{klein_second_1986}.\is{second language acquisition} 



The kinds of causal processes seen in the ontogenetic frame include strategies 
for learning and motivations for learning. Some of these strategies and 
motivations can be complementary, and some may be employed at distinct 
phases of life. Causal processes involved in this frame include 
conditioning, statistical learning and associated mechanisms like 
entrenchment and pre-emption \citep{tomasello_constructing_2003}, adaptive docility\is{docility} \citep{simon_mechanism_1990}, a pedagogical stance \citep{gergely_sylvias_2006}, and long-term 
memory \citep{kandel2009biology}.



\subsection{Phylogenetic (biological evolution)}
\is{phylogenetic frame}


In a phylogenetic frame we ask how our species first became able to learn and use language. This is part of a broader set of questions about the 
biological evolution and origin of humankind.\is{evolution!biological} It is a difficult topic to 
study, but this has not stopped a vibrant bunch of researchers from 
making progress \citep{hurford_origins_2007,hurford_origins_2012,levinson_language_2014}.\is{evolution!of language capacity} 



Causal processes in a phylogenetic frame include those typically described in evolutionary biology. They invoke 
concepts like survival, fitness, and reproduction of biological 
organisms \citep{ridley_evolution_1997,ridley_evolution_2004}, which in the case of language means 
members of our species. The basic elements of 
Darwinian natural selection\is{selection} are essential here: competition among individuals in a 
population, consequential variation in individual characteristics, 
heritability of those characteristics, exaptation, non-telic design, and so forth \citep{darwin_origin_1859,dawkins_selfish_1976,jacob_evolution_1977,mayr_growth_1982}.



\subsection{Enchronic (social interactional)}
\is{enchronic frame}


In an enchronic frame, we look at language in the context of social interaction.\is{social interaction} When we communicate, we use sequences of moves made up of speech, gesture,\is{gesture} and other kinds of signs. The causal processes of interest involve structural relations of sequence 
organization (practices of turn-taking and repair which organize our 
interactions; \citealt{schegloff_sequencing_1968,schegloff_sequence_2007,sacks_simplest_1974,schegloff_preference_1977,sidnell_handbook_2012}) and ritual or affiliational relations of 
appropriateness, effectiveness, and social accountability \citep{heritage_garfinkel_1984,atkinson_structures_1984,stivers_morality_2011,enfield_relationship_2013}. 



Turn-taking\is{turn-taking} in conversation\is{conversation} operates in the enchronic 
frame, as do speech act\is{speech acts} sequences such as question-answer, 
request-compliance, assessment-agreement, and suchlike (see \citealt{enfield_language_2014}). Enchronic processes tend to take place at a temporal 
granularity around one second, ranging from fractions of seconds up to 
a few seconds and minutes (though as stressed here, time units are not 
the definitive measure; exchanges made using email or surface mail may stretch out over much greater lengths of time). 



Enchronic processes and structures are the focus in 
conversation analysis and other traditions of research on communicative 
interaction. Some key causal elements in this frame include relevance \citep{garfinkel_studies_1967,grice_logic_1975,dan_sperber_relevance:_1995}, local motives \citep{schutz_phenomenology_1970,leontev_problems_1981,heritage_garfinkel_1984}, sign-interpretant relations (\citealt{kockelman_semiotic_2005,kockelman_agent_2013}, \citealt[Chapter 4]{enfield_relationship_2013}), and social accountability \citep{garfinkel_studies_1967,heritage_garfinkel_1984}.



\subsection{Diachronic (social/cultural history)}
\is{diachronic frame}


In a diachronic frame, we look at elements of language as historically 
conventionalized patterns of knowledge and/or behavior. If the question 
is why a certain linguistic structure is the way it is, a diachronic 
frame looks for answers in processes that operate in historical 
communities. While of course language change has to be actuated\is{actuation} at a 
micro level \citep{weinreich_empirical_1968,labov_mechanism_1986,eckert_linguistic_2000}, for a linguistic item to be found in a language, that 
item has to have been diffused\is{diffusion} and adopted throughout a community 
before it can have become a convention. 



Among the causal processes of interest in a diachronic frame are the 
adoption and diffusion of innovations, and the demographic ecology that 
supports cultural transmission \citep{rogers_diffusion_2003}. Population-level 
transmission is modulated by microgenetic\is{microgenetic frame} processes of extension, inference, and reanalysis that feed grammaticalization\is{grammaticalization} 
\citep{hopper_grammaticalization_1993}. 



Of central importance in a diachronic frame are social processes of 
group fission and fusion \citep{aureli_fission-fusion_2008}, migration \citep{manning_migration_2005}, and sociopolitical relations through history \citep{smith_inquiry_1776,marx_german_1947,runciman_theory_2009}. The timescales of interest in a 
diachronic frame are often stated in terms of years, decades, and 
centuries.



\subsection{Synchronic (representation of relations)}
\is{synchronic frame}


Finally, a synchronic frame is different from the other frames mentioned 
so far because time is removed from consideration, or at least 
theoretically so. One might ask if it is a causal frame at all. But 
if we think of a synchronic system as a true description of the items 
and relations in a person's head, as coded, for example, in their 
memory, then this frame is real and relevant, with causal implications, 
even if we see it as an abstraction (e.g., as bracketing out 
near-invisible processes that take place in the fastest levels of 
Newell's ``biological band''; see section \ref{Distinctframesandforces}, above). 



In Saussure's famous comparison, language is like a game of chess. If we 
look at the state of the game half way, a diachronic\is{diachronic frame} frame 
would view the layout in terms of the moves that had been made up 
to that point, and that had created what we now see. A 
synchronic account would do no more than describe the positions and 
interrelations of the pieces on the board at that point in time. For an 
adequate synchronic description, one does not need to know how the set 
of relations came to be the way it is. 



There are two ways to take this. One is to see the synchronic frame as a purely 
methodological move, an abstraction that allows the professional 
linguist to describe a language as a whole system that hangs together. 
Another -- not in conflict with the first -- is to see the 
synchronic description of a language as a hypothesis about what is 
represented in the mind of somebody who knows the language. 



A synchronic system cannot be an entirely atemporal concept. At the very 
least this is because synchronic structures cannot be inferred without 
procedures that require time; e.g., the enchronic\is{enchronic frame} sequences that we use
in linguistic elicitation with native speaker informants. But a synchronic system is 
clearly distinct from an associated set of ontogenetic processes,\is{ontogenetic frame} on the one hand, 
and diachronic\is{diachronic frame} processes, on the other (though it is causally 
implied in both). We can infer an adult's knowledge of language and 
distinguish this from processes including the learning that led to this 
knowledge and the history that created the conventional model for 
this knowledge (but which neither the learner nor the competent speaker 
need have had access to). 



The goal here is to define frames that are relevant to a natural, causal account of language. So when I talk about a 
synchronic frame I mean a way of thinking about the conceptual representations of a language that make it possible for people to produce and interpret utterances in that language. 



Causality in a synchronic frame is tied to events that led \textit{to} 
the knowledge, and to events that may lead \textit{from} it, as well as how the 
nature and value of one convention may be dependent on the nature and 
value of other conventions that co-exist as elements of the same system.



\section{Interrelatedness of the frames}



How are these frames interrelated? As \citet[276]{raczaszek-leonardi_multiple_2010} says, ``even if a researcher aims to focus on a particular scale and system, he 
or she has to be aware of the fact that it is embedded in others''. Other 
authors (\citealt[179]{cole_cultural_1996}, \citealt[192]{macwhinney_emergence_2005}) have asked: What are the 
forces that cause these frames to ``interanimate'' or ``mesh''? The way to find out would be to test and extend the useful suggestions of authors like \citet{newell_unified_1990}, \citet[184--185]{cole_cultural_1996}, \citet{macwhinney_emergence_2005}, \citet[279--286]{lemke_across_2000} and \citet{uryu_ecology_2014}. 



How might the outputs of processes foregrounded within any one of these 
explanatory frames serve as inputs for processes foregrounded within any 
of the others? Answers to this question will greatly enrich our tools for 
explanation.



\section{The case of Zipf's length-frequency rule}
\label{zipflengthrule}
\is{length-frequency rule}
Why is it good to have a set of distinct causal frames for language?
Because it offers explanatory power. Consider the observation made by Zipf that \textquoteleft every language shows an inverse relationship between the lengths 
and frequencies of usage of its words' \citep[66]{zipf_human_1949}.\footnote{I am grateful to Martin Haspelmath for insisting on the distinction between Zipf's Law and Zipf's length-frequency rule (cf. \citealt{NewmanPowerLaws2006}). Zipf's Law states that there is a correlation between the frequency of an item and its frequency rank relative to other items in a set. His length-frequency rule states that the shorter a word is, the more frequently the word is used.} Zipf suggested that the correlation between word length and frequency is explained by a psychological preference for minimizing 
effort.\is{minimal effort principle} If we take this as a claim that synchronic structures in 
language are caused by something psychological -- though Zipf's own claims 
were rather more nuanced -- this raises a linkage problem\is{linkage problem} (\citealt[201]{clark_psychological_1984}). 



The problem is that a person's desire to minimize effort cannot directly affect a synchronic system's structure. 
A cognitive preference is a property of an individual, while a synchronic 
fact is shared throughout a population. Something must link the two. 
While it may be true that the relative length of the words I know 
correlates with the relative frequency of those words, this fact was 
already true of my language before I was born. The correlation cannot have been caused by my cognitive 
preferences. How, then, can the idea be explicated in causal terms?



As was clear to \citet{zipf_human_1949}, to solve this problem we appeal to 
multiple causal frames.\is{causal frames} We can begin by bringing diachronic\is{diachronic frame} processes 
into our reasoning. A presumption behind an account like Zipf's is 
that all members of a population have effectively the same biases.\is{cognitive biases} The key to understanding the status of a microgenetic\is{microgenetic frame} bias like 
``minimize effort in processing where possible'' is to realize that this 
cognitive tendency has an effect only in its role as a \textit{transmission bias}\is{transmission biases} in a diachronic process of diffusion\is{diffusion} of convention 
in a historical population (see below chapters for explication of 
diachrony as an epidemiological\is{epidemiology} process of biased transmission, 
following Rogers \citeyear{rogers_diffusion_2003}, Sperber \citeyear{sperber_anthropology_1985}, and Boyd and Richerson \citeyear{boyd_culture_1985,boyd_origin_2005}). The synchronic\is{synchronic frame} facts are an aggregate outcome of individual people's biases multiplied in a community and through time. The bias has a 
causal effect precisely in so far as it affects the likelihood that a
pattern will spread throughout that community. 



Now, while the spread of a pattern and its maintenance as a convention in a group are diachronic\is{diachronic frame} processes, a transmission bias can operate in three other frames. In an ontogenetic frame,\is{ontogenetic frame} a correlation between the shortness of words and the frequency of words might make the system easier to learn. This bias causes the correlation to become more widely distributed in the population. In a microgenetic frame,\is{microgenetic frame} people may want to save energy by shortening a word that they say often, again broadening the distribution of the correlation. And an enchronic frame\is{enchronic frame} will capture the fact that communicative behavior is not only regimented by individual-centered biases in learning, processing, and action, but also by the need to be successfully understood by another person if one's communicative action is going to have its desired effect. The presence of another person, who displays their understanding, or failure thereof, in a next move -- criterial to the enchronic frame -- provides a selectional counter-pressure against people's tendency to minimize effort\is{minimal effort principle} in communicative behavior. One's action has to be recognized by another person if that action\is{social action} is going to succeed (\citealt[21]{zipf_human_1949}, \citealt[Chapter 9]{enfield_relationship_2013}).



If we adopt a rich notion of a diachronic frame in which transmission 
biases play a central causal role, we can incorporate the ontogenetic, 
microgenetic and enchronic frames in explaining synchronic facts. We do this by invoking the mechanisms of \textit{guided variation}\is{guided variation} explicated by 
\citet{boyd_culture_1985,boyd_origin_2005} and explored in subsequent work by 
others \citep{kirby_function_1999,kirby_ug_2004,christiansen_language_2008,chater_language_2010}. This allows us to hold 
onto Zipf's insight, along with similar claims by authors such as 
Sapir before him, and Greenberg after him, who both also saw connections 
between individual-level psychological biases\is{cognitive biases} and community-level synchronic facts. \citet{greenberg_universals_1966} implied, for example, that there is a kind of cognitive 
harmony in having analogous structures in different parts of a language 
system. \citet[154--158]{sapir_language:_1921} suggested that change in linguistic 
systems by drift\is{drift} can cause imbalances and \textquoteleft psychological 
shakiness', which motivates the reorganization of grammar to avoid 
that mental discomfort. 



Similar ideas can be found in work on grammaticalization\is{grammaticalization} \citep{givon_syntax:_1984,bybee_language_2010} and language change\is{language change} due to social contact\is{language contact} \citep{weinreich_languages_1953}, leading to the same conclusion: Synchronic patterns can have 
psychological explanations but only when mediated by the aggregating 
force of diachronic processes.\is{diachronic frame} 



The point is central to explaining other observed 
correlations in language and its usage, for example that more frequent words change more slowly 
\citep{pagel_frequency_2007}, that differences in processes of 
attention and reasoning correlate with differences in the grammar of the 
language one speaks \citep{whorf_language_1956,lucy_language_1992,slobin_thought_1996}, 
that ways of responding in conversation\is{conversation} can be constrained by collateral 
effects\is{collateral 
effects} of language-specific grammatical structures \citep{sidnell_language_2012}, that tendencies in natural meaning\is{natural meaning} can correlate with universals\is{universals} in the sounds of words \citep{dingemanse_is_2013}, and that cultural values\is{cultural values} can shape grammatical categories \citep{hale_notes_1986,wierzbicka_semantics_1992,chafe_loci_2000,enfield_ethnosyntax:_2002,everett_cultural_2005,everett_language:_2012}. But most if not all of these claims bracket out some elements of the full 
causal chain involved. To give a complete and explicit account, multiple frames are needed.





 \newpage