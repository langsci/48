
\chapter{The Micro/Macro Solution}
\label{micromacrosolution}

Do cultural totalities exist?\is{cultural totalities} As members of a group we may feel certain that 
there is a cultural totality around us. But we never directly observe it. As 
\citet[56]{fortes_social_1949} put it:

\begin{quotation}
Structure is not immediately visible in the \textquoteleft concrete 
reality'. It is discovered by comparison, induction and analysis based 
on a sample of actual social happenings in which the institution, 
organization, usage etc. with which we are concerned appears in a 
variety of contexts.
\end{quotation}

This mode of discovery is not only used by ethnographers who are studying culture.\is{culture} It is also used by children\is{children} whose task is to become competent adults (see \citealt{brown_language_2014}). 



If our experience of culture is in the micro, how do we 
extrapolate to the macro?\is{micro-macro relation} When Parry and Bloch wrote about money\is{money} 
and its status, they stressed that there are local differences between cultures and the 
effects on the meaning that money comes to have. But they also acknowledged a 
certain unity across cultures: 

\begin{quotation}
[This unity is] neither in the meanings 
attributed to money nor in the moral evaluation of particular types of 
exchange, but rather in the way the totality of transactions form a 
general pattern which is part of the reproduction of social and 
ideological systems concerned with a time-scale far longer than the 
individual human life. \citep[1]{parry_money_1989} 
\end{quotation}



In terms that apply more generally to the micro/macro issue, there is \textquoteleft something very general about the relationship between the 
transient individual and the enduring social order which transcends the 
individual' \citep[2]{parry_money_1989}. It brings to mind Adam Smith's (\citeyear[book 4, ch. 2]{smith_inquiry_1776}) discussion of the relation between the 
motivations of individuals and the not-necessarily-intended 
community-level aggregate effects of their behavior \citep{schelling_micromotives_1978,rogers_diffusion_2003,hedstrom_social_1998}. \citet[29]{parry_money_1989} contrast \textquoteleft short-term order' with \textquoteleft long-term reproduction', and they suggest that the two must be linked. 



This brings us back to the transmission criterion,\is{transmission criterion} an idea that will help to bridge the micro/macro divide. If a person is to function as a member of a social group, he or she needs to individually construct, in the ontogenetic frame,\is{ontogenetic frame} the ability to produce and properly interpret the normative behaviour of others.\is{normative behaviour} 



Not even a cultural totality is exempt from the transmission criterion. 
Individual people have to learn the component parts of a totality during 
their lifetimes (in ontogeny),\is{ontogenetic frame} and they must be motivated to reproduce\is{reproduction} the behaviours 
(in microgeny\is{microgenetic frame} and enchrony\is{enchronic frame}) that stabilize the totality and cause it to endure beyond their own 
lives and lifetimes (in diachrony)\is{diachronic frame}. A person's motivation can be in the form of a
salient external pressure such as the threat of state violence.\is{state violence} But it usually comes from the less visible force of normative accountability\is{normative accountability} \citep{heritage_garfinkel_1984,enfield_relationship_2013}. 



In the social/cultural contexts of our daily lives, everything we do will be interpreted as meaningful. \textquoteleft The big question is not whether 
actors understand each other or not', wrote \citet[367]{garfinkel_perception_1952}. \textquoteleft The fact is that they do understand each other, 
that they \textit{will} understand each other, but the catch is that 
they will understand each other regardless of how they \textit{would} 
be understood.' This means that if you are a member of a social group, you are not 
exempt from having others take your actions to have meanings, whether or 
not these were the meanings you wanted your actions to have. 



As \citet[321]{levinson_pragmatics_1983} phrases it, also echoing Goffman and Sacks, we 
are \textquoteleft not so much constrained by rules or sanctions, as caught up in a 
web of inferences'.\is{web of inferences} We will be held to account for others' 
interpretations of our behavior and we know this whether we like it or 
not.\footnote{This does not mean that we are accountable for just any interpretation, but only those interpretations that are grounded in social norms. For example, if you are in the habit of going barefoot on the street, you can expect people to draw attention to this whether you like it or not (in a way that they will not if you are in the habit of wearing shoes).} This is a powerful force in getting us to conform. Accountability 
to norms \textquoteleft constitutes the foundation of socially organized conduct as a 
self-producing environment of \textquotedblleft perceivedly normal" activities' 
\citep[119]{heritage_garfinkel_1984}. The thing that tells us what counts as normal is of 
course the culture.\is{culture} 



\begin{quotation}
With respect to the production of normatively appropriate conduct,\is{normative behaviour} all 
that is required is that the actors have, and attribute to one another, 
a reflexive awareness of the normative accountability\is{normative accountability} of their actions. 
For actors who, under these conditions, calculate the consequences of 
their actions in reflexively transforming the circumstances and 
relationships in which they find themselves, will routinely find that 
their interests are well served by normatively appropriate conduct. With 
respect to the anarchy of interests, the choice is not between 
normatively organized co-operative conduct and the disorganized pursuit 
of interests. Rather, normative accountability is the \textquoteleft grid' by 
reference to which \textit{whatever} is done will become visible and 
assessable. \citep[117]{heritage_garfinkel_1984}
\end{quotation}



One might ask what is \textquoteleft normatively appropriate conduct'. The answer must 
include any of the kinds of behaviours discussed in the above section on 
cultural systems: for example, behaving in accordance with the rules of 
a section\is{sections} system by marrying someone of the right category (or being 
able to give reasons why you have done otherwise). They would 
not be cultural behaviours if they were not regimented in a community by 
accountability to norms (and probably also laws). 



So the path that is both the least resistant and the most empowering for 
a person is to learn the system that generates a shared set of 
normative interpretations of people's behaviour, and then go with the 
flow. This is how the totality cannot exist without the individuals, 
while --- paradoxically --- appearing to do just that. We create and maintain the very systems that constrain us.



The close relationship between short-term order and long-term 
reproduction is an asymmetrical one. Short-term order is where the 
causal locus of transmission is found. It is where acceleration, 
deceleration, and transformation in cultural transmission occurs 
\citep{schelling_micromotives_1978,rogers_diffusion_2003,sperber_anthropology_1985,sperber_explaining_1996}. 



From all of this it is clear that cultural systems exist and they both 
constrain us and guide us. The question is: How are systems transmitted? The regulation of individual behaviour in the cultural 
totality is not achieved by mere emergence.\is{emergence} It is not like the self-oriented behaviour of a bird 
in the seemingly concerted movement of a flock. Individuals' behaviour 
is regulated by norms, in an effectively telic way. A good deal of 
cultural regimentation is done through explicit instruction, often with 
reference to norms, and sometimes with reference to punishable laws. 



To see how whole cultural systems are transmitted, we have to draw on item-based processes of transmission. As we 
saw in the last chapter, the only good causal account we have for social 
transmission through populations and across generations is one that 
works in terms of items, not whole systems.



\section{The combinatoric nature of cultural items in general}


Recall that the context bias\is{transmission bias!context bias} is grounded in the fact that one cannot 
behold any so-called item without beholding it \textit{in relation to}
something else, including not only things of similar kinds, but also the social norms and intentions associated with items and 
the contexts in which they appear. So, I cannot know what a hammer is if I do not see it in relation to the human body, 
timber and nails, people's intentions to build things, conventional 
techniques for construction, and so on. 



These relations --- which themselves are interrelated --- form an indispensible 
part of what I am referring to by the term \textit{item}.\is{linguistic item} When a cultural item 
diffuses, what is diffusing is something less like an object 
and more like a combinatoric relation. So, a hammer incorporates a 
handle or grip. The handle or grip has a combinatoric relation to the 
human hand insofar as the handle and the hand are practically and 
normatively designed to go together. The handle is designed that way 
because of how the human hand is. The handle only makes sense in terms 
of a person's hand. 



This going together of the handle and the human hand is like a grammatical 
rule. In a similar way, the handle of the hammer and the \textit{head} of the 
hammer go together both 
practically and normatively. The head of the hammer, in turn, goes with a nail. The nail, in turn, goes with timber, and so forth. So we see how the cultural items that diffuse in communities necessarily 
incorporate --- and advertise --- their rules of fit with other items. 



The sprawling yet structured systems\is{systems} that we call languages\is{languages, as unit of analysis} have the same kinds of properties of incorporation\is{incorporation} and contextualization\is{contextualization} that I 
have just described for concrete objects. So, if speakers of a 
language have borrowed\is{borrowing} a word from another language, this does not mean they have 
merely adopted a pairing of sound and concept. They must also have adopted a way of relating the word to their existing language system 
(whether or not this relation resembles the one used in the source 
system). 



The word will not be usable if it does not have combinatoric properties that specify how it fits with other words. 
The norms for combining the word in usage may be borrowed along with the 
word itself, or they may be provided by existing structures in the borrowing 
language, or they may even be innovated in the process of incorporation.\is{incorporation} 



The combinatoric relations surrounding a cultural item 
do not have to diffuse along with that item. But a cultural 
element must have \textit{some} 
combinatoric relation to other cultural items in the same domain if it is to function and circulate. That relation can just as well be invented by the people who adopt the item, in line with the contraints of their own culture and world view. This 
is the point that authors like Sahlins and Eckert, mentioned above, 
have stressed for culture.



So, structuralist linguists like \citet[1]{donegan_rhythm_1983} are right when they 
say that a language \textquoteleft is not just a collection of autonomous parts'. But
this does not mean that a language is \textquoteleft a self-contained 
whole'. The same applies when cultural anthropologists 
refer to the \textquoteleft cultural totality'.\is{cultural totalities} 



We never encounter whole systems except one fragment at a time, in microgeny\is{microgenetic frame} and enchrony.\is{enchronic frame} Our 
\textquoteleft partial experience' \citep[191]{le_page_acts_1985} is not 
experience of the whole system. But nor is it experience of stand-alone 
items. When we experience culture, we experience meaningful items in 
relations of functional incorporation\is{incorporation} and contextualization\is{contextualization} with other 
such items. 


Each such relation is, effectively, a combinatoric principle, like a 
norm for forming a grammatical sentence or for using a hammer and nail 
in the appropriate way. These relations are at the centre of the 
framework being proposed here. These relations are what is transmitted. 
They have an inherent connection to a cultural system or field, 
but this system or field has no pre-given size or outer borderline.



\citet{bloch_well-disposed_2000} says that old critiques of diffusionism\is{diffusionism} in anthropology 
also work as critiques of today's item-based accounts. I would say that 
the problems are handled by the simple conceptual shift 
being proposed here. The relevant unit of cultural transmission (meme\is{memes} or 
whatever) is not \textit{a piece}. The relevant unit is \textit{a piece and its 
functional relation to a context}. This might seem obvious. But when we 
make it explicit, the fear of a disembodied view of cultural units goes 
away. The required conceptual move is not to take 
items and put them in a context. Their relation to a context is what \textit{defines} them. 



\section{Solving the item/system problem in language}


Identifying the \textit{relation to context} as the common unit of 
analysis of both items and systems is necessary but not yet sufficient. We 
need an account of how this scales up into large structured sets of such 
relations. Let us consider the question in connection to language. Every linguistic convention in a community is a 
product of general mechanisms of social diffusion.\is{diffusion} Each convention has its own history. Every word, 
every morpheme, every construction has followed its own historical path to 
community-level acceptance. As \citet[444]{bloomfield_language_1933} said, \textquoteleft individual forms may have had very 
different adventures'. 



This does not mean languages are mere bundles 
of items. They are large, structured, systematic wholes. Psychologically, languages exist in people's minds and bodies.
They take the form of idiolects.\is{idiolects} Intersubjectively, 
languages exist at a \textit{community} level to the extent that people's 
idiolects are effectively alike in structure and content, as 
demonstrated by the evidently tolerable degree of success 
of communication \citep{enfield_utility_2015}. We can now 
specify some forces that bring items together and structure them into 
systems.



\section{Centripetal and systematizing forces}

\is{centripetal forces on items}
When we say that two people speak the same language, we mean that two individuals' knowledge of a language system --- synchronically,\is{synchronic frame} as can be seen in their enchronic\is{enchronic frame} and microgenetic\is{microgenetic frame} behaviour --- is effectively (though never exactly) shared. This sharedness exists because a large number of the same linguistic variants have been channelled, in a huge set, along the same historical pathways. This gives the impression that a language is passed down as a whole, transcending lifetime after lifetime of the individuals who learn and embody the system. 



This is the point made by \citet{thomason_language_1988}: Normal social 
conditions enable children, as first language learners,\is{first language acquisition} to construct 
idiolects that effectively match the idiolects of the people they learn 
from --- i.e., those with whom children share a household and an immediate social 
environment, and, incidentally, who are most likely to share their 
genes. Normal transmission\is{normal transmission} is what allows historical linguists\is{historical linguistics} to 
abstract from the fact that each linguistic variant has its own career, 
and in turn to treat the whole language as having one 
spatial-historical trajectory.



In many cases this is a reasonable and successful methodological 
presumption \citep{haspelmath_how_2004}. But in 
situations other than those of normal transmission\is{normal transmission} \citep{le_page_acts_1985,thomason_language_1988}, linguistic items do 
not always travel together, but may follow separate paths, making 
visible what is always true but usually obscured by items' common 
destiny in practice, namely: Each item has its own history. 



Genealogical continuity in language change\is{language change} is typically taken to be the 
norm. Whenever we see that linguistic systems\is{systems} are permeable, for instance in certain language contact situations where the 
components of languages are prised apart, special explanations are 
demanded. 

\section{On normal transmission}

To say that a child inherits a language from her parents is a misleading representation of what happens in language acquisition.\is{first language acquisition}\is{ontogenetic frame} The idiolect of the child is not acquired like DNA in a bundle.\is{genetics} Patterns of constituency and grammatical relations do not unfold in children like the shapes of their bodily organs. Through practice, children have to learn, construct, and maintain skills and ideas for ways of saying things. 

\begin{quotation}
The \textquoteleft rules' of a child's \textquoteleft native language' ... are in any case likely to be tentative hypotheses, easily modified by fresh semantic needs, fresh contacts, fresh analogies. \textquoteleft Syntax' in the grammarian's sense is what emerges from this process, not what it starts from. \citep[190]{le_page_acts_1985} Logic and universal grammar, then, are targets towards which, rather than the starting point from which, human linguistic activity proceeds. The origins of that activity are like those of a game which gradually develops among players, each of whom can experiment with changes of the rules, all of whom are umpires judging whether new rules are acceptable. \citep[197]{le_page_acts_1985} 
\end{quotation}

This transmission takes place through air, over days, weeks, months, years, with interference and noise. Every bit of the idiolect's structure has to be passed over and constructed from scratch by the learner. This task is made possible by the sheer deluge of linguistic data --- a \textit{Niagara of words}, as \citet[12ff]{hayakawa_language_1978} called it --- which people are exposed to, and produce in turn. Child language acquisition is a process of building \citep{tomasello_constructing_2003}, resulting in something like a grammatical totality in the child's competence.\is{competence} But whatever totality a person has built, it is instantiated somehow in the head and so (a) will never go public as a whole and (b) will be destroyed when the person dies. The system is neither observed nor passed on as a whole unit, only ever fragment-by-fragment.

\citet{dunbar_grooming_1996} has hypothesized that prelinguistic human ancestors created language\is{evolution!of language} as a way to lessen time pressure due to the need to manage an expanding number of social associates. Sustaining a social network\is{social networks} by means of linguistic contact is time-consuming. Where personal exchange or strong network ties are involved, we are necessarily oriented towards a limited group. The size of networks is constrained by the time it takes to maintain these relationships. However, the number of non-personal exposure ties --- passive seeing and hearing, especially due to media and high population density --- is potentially massive. The invention of writing has drastically changed the proportion of personal and non-personal sources of exposure to innovation.

\citet{thomason_language_1988} invoke an idea of \textit{normal transmission} (see above). They define normal transmission\is{normal transmission} \textquoteleft by exclusion' \citep[10]{thomason_language_1988}, in terms of how \textquoteleft perfectly' all sub-systems of a language are reproduced in children's idiolects. In normal transmission, linguistic input from outgroup people has negligible impact on a child's construction of an idiolect highly convergent with the idiolects of the parents' generation. 
Normal transmission, in Thomason and Kaufman's sense, is a social fact \citep[12]{thomason_language_1988}, though it is defined by formal facts about child language acquisition in a community:

\begin{quotation}
[A] claim of genetic relationship [between a \textquoteleft parent' and a \textquoteleft daughter' language] entails systematic correspondences in all parts of the language because that is what results from normal transmission: what is transmitted is an entire language --- that is, a complex set of interrelated lexical, phonological, morphosyntactic, and semantic structures. \citep[11]{thomason_language_1988}
\end{quotation}
	
Here is how I understand Thomason and Kaufman's point. To say that a \textquoteleft genetic' relationship holds between parent and daughter languages is to use a metaphor, and to use this metaphor is harmless as long as the older generation's idiolects are reproduced so closely in the idiolects of the younger generation that it is \textit{as if} the new idiolects were replicas of the old. This is effectively what happens in the case of normal transmission. There is a relentless and focussed linguistic sign deluge from people of the learner's own group.



But another question remains. How can we explain the relative impermeability of linguistic systems in circumstances of normal transmission? Stability in conventional systems\is{systems} is no less in need of 
explanation than variation or change \citep{bourdieu_outline_1977,sperber_explaining_1996,sperber_cognitive_2004}. What are the forces that cause linguistic variants to follow en masse a single path of diffusion 
and circulation, and to hold together as structured systems? Let us briefly 
consider three such forces.



\subsection{Sociometric closure}
\label{sociometricclosure}

\is{sociometric closure}

A first centripetal force is sociometric closure. This arises from a trade-off between strength and number of relationship ties in a 
social network.\is{social networks} If a person is going to maintain a social relationship, she 
has to commit a certain amount of time to this. Time is a 
finite resource. This puts a structural constraint on the possible 
number of relationships one can maintain \citep{hill_social_2003}. 
The result is a relatively closed circulation of currency within a 
social economy of linguistic items. It causes people's inventories of 
items (i.e., their vocabularies, etc.) to overlap significantly, or to 
be effectively identical, within social networks.



This helps to account for how people who interact often
can have a common set of variants. It does not account for the 
system-like nature of the relations among those items. We 
turn now to two forces of systematization inherent 
to grammar, in the paradigmatic\is{paradigmatic axis} and syntagmatic axes.\is{syntagmatic axis}



\subsection{Trade-off effects}
\label{tradeoff}

\is{trade-off effects}
One systematizing force comes from functional trade-off effects that 
arise when a goal-oriented person has alternative means to similar ends. When different items come to be used 
in a single functional domain, those items can become formally and 
structurally affected by their relative status in the set. This happens 
because the items compete for a single resource, namely, our selection of 
them as means for our communicative ends. 



When \citet[19ff]{zipf_human_1949} undertook \textquoteleft a study of human speech as a set of 
tools', he compared the words of a language with the tools in an 
artisan's workshop. Different items have different functions, and different relative functional loads. In a vocabulary, \citet[21]{zipf_human_1949} 
argued, there is an internal economy of words, with trade-offs 
that result in system effects like the observed correlation between 
the length of a word (relative to other words) and the frequency of use 
of the word (relative to that of other words).\is{length-frequency rule} 



Zipf reasoned that \textquoteleft the more frequent tools will tend to be the lighter, 
smaller, older, more versatile tools, and also the tools that are more 
thoroughly integrated with the action of other tools' \citep[73]{zipf_human_1949}. He 
showed that the more we regard a set of available means as alternatives 
to each other in a functional domain, the more they become defined in 
terms of each other, acquiring new characteristics as a result of their 
role in the economy they operate in. In other words: The more we 
treat a set of items as a system, the more it becomes a system.\is{systems}



\subsection{Item-utterance fit, aka content-frame fit}
\label{itemutterance}
\is{item-utterance fit}\is{content-frame fit}

A final key source of grammatical structure is grammatical 
structure itself. The \textit{utterance }is a core structural locus in 
language. An utterance is a local context for the interpretation of a 
linguistic item. It is an essential ratchet between item 
and system. As Kirby writes, although \textquoteleft semantic information' is what 
linguistic utterances most obviously convey, \textquoteleft there is another kind of 
information that can be conveyed by any linguistic production, and that 
is information about the linguistic system itself' \citep[123]{kirby_transitions:_2013}: 

\begin{quotation}
When I produce the sentence \textquoteleft these berries are good' I may 
be propagating cultural information about the edibility of items in the 
environment via the content of the sentence. At the same time I may also 
be propagating information about the construction of sentences in my 
language.
\end{quotation}
%(Kirby 2013, 123). 

In this way, an utterance is a frame and a vehicle for replicating linguistic variants \citep{croft_explaining_2000}. 



Item-utterance fit is the structural fit between diffusible types of 
linguistic item and the token utterances in which they appear. It is an 
instance of the more general \textit{content-frame} schema \citep{levelt_speaking:_1989} also 
observed in phonology (MacNeilage \citeyear{macneilage_frame/content_1998}; see \citealt[54-55]{enfield_relationship_2013}), and a 
case of the \textquoteleft functional relation to context' defined above as a common property of items and systems. Now we see that it is not just a 
common property. It is the very property that connects items with systems. An utterance is an incorporating\is{incorporation} and contextualizing\is{contextualization} 
frame for the diffusion of replicable linguistic items,\is{linguistic items} \textit{and} it is a frame for the diffusion of the combinatoric rules from which the 
higher-level system\is{systems} is built. 



\section{A solution to the item/system problem?}


The above considerations suggest that the item/system problem\is{item/system problem} can be 
solved if the following three forces apply in the 
biased transmission of cultural items: 

\begin{enumerate}

\item {\textit{Congregation:} Items are brought together and `bundled' by the population-level effects of inward-directed sociometric biases.}

\item {\textit{Specialization:} Items then effectively compete for selection in the same functional contexts, and come to be specialized as alternative means for related functional ends.}
 \item {\textit{Combination:} Items in a set come to combine with each other in functional ways, via context biases and the relation of item-utterance fit.} 

\end{enumerate}

We can expect there to be analogous relations to 
item-utterance fit  (=content-frame fit) in the domain of culture. Think, for instance, of 
systems of social relations in kinship,\is{kinship systems} or systems of material culture 
and technology in households and villages. 



Zipf's (\citeyear{zipf_human_1949}) analogy is useful here. For his \textquoteleft economy of tools-for-jobs 
and jobs-for-tools' to get off the ground, one first needs a \textit{workshop}, somewhere the set of tools is assembled in one place, 
and made accessible to a person with goals. In language and 
culture, this is achieved by sociometric closure\is{sociometric closure} (\ref{sociometricclosure}, above): the more you talk with certain people, the more ways of talking you will share with these people. 
Then, one works with the set of tools, using them as alternative specialized means to similar or related ends (\ref{tradeoff}, above). Finally, these tools will, whether 
by design or by nature, enter into the relations of incorporation\is{incorporation} and 
contextualization\is{contextualization} that define their both their functional potential and their system status (\ref{itemutterance}, 
above). 



Now this should look familiar to the linguist. Once we get an \textit{inventory} or lexicon of items that have specalized functions within a given domain, they will naturally enter into the 
\textit{paradigmatic}\is{paradigmatic axis} and \textit{syntagmatic}\is{syntagmatic axis} relations that 
define semiotic systems\is{systems} in the classical sense.









\newpage