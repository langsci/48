\chapter{Causal Dynamics }





\section{Perspectives on language}


Consider some of the different types of process that one might focus on in studying language. There are the finely-timed perceptual, cognitive, and motoric processes involved in producing and comprehending language. There are the early lifespan processes by which children acquire linguistic and communicative knowledge and skills, and the evolutionary processes that led to the unique emergence of the requisite cognitive capacities for language in our species. There are the ways in which linguistic utterances are involved in the sequential interlocking of social actions. There are the processes and products of language change, with links between processes at historical time-scales and evolutionary time-scales. There is linguistic variation and its role in how historical change in language is socially conducted in human populations. And there are phenomena that can be described without reference to process or causation at all, as seen in linguistic grammars, dictionaries, ethnographies, and typologies, where relationships rather than processes tend to be the focus.\footnote{On language production and comprehension, see for example Levelt 1989, 2012; Emmorey 2002; McNeill 2005; Cutler 2012. On language acquisition see Brown and  Gaskins, 2014; Schieffelin and Ochs 1986; Tomasello 2003. On language evolution see Hurford 2007, 2012;  Tomasello 2008; Hauser, Chomsky, and Fitch 2002; Chomsky 2011. On conversational sequence, see Schegloff 1968; Goffman  1981a; Goodwin 2000, 2006; Sidnell and Stivers 2012; Enfield 2013;  Enfield and Sidnell 2014; Sidnell and Enfield 2014. On language change, see Dixon  1997; Harris and Campbell 1995; Hopper and Traugott 1993; Hanks 2010; for evolutionary perspectives, see Boyd and Richerson 1985, 2005; Durham 1991; Smith, Brighton, and Kirby 2003; for the role of variation in change, see Labov 2011;  Trudgill 2010; Eckert 2000.}

These different kinds of foci correspond roughly with distinct research 
perspectives. But they do not merely represent disciplinary 
alternatives. The different perspectives suggested above can be seen to 
fit together as parts of a larger conceptual apparatus. 



To sketch that apparatus, I define a set of six frames that I suggest 
are useful for orienting our work, and that should remind us of the 
perspectives that are always available and relevant, but that we might 
not be focusing on. They do not constitute a definitive set of 
frames-there is no definitive set-but I suggest that they correspond 
well to the most important causal domains, that they conveniently group 
similar or tightly interconnected sets of causal mechanism under single 
rubrics, and that together they cover most of what we need for providing 
answers to our questions in research on language. 



The frames are called \textit{Microgenetic}, \textit{Ontogenetic}, 
\textit{Phylogenetic}, \textit{Enchronic}, \textit{Diachronic}, 
and \textit{Synchronic}; or M.O.P.E.D.S. for short. While such frames 
are sometimes referred to as \textquoteleft timescales', the differences between them 
are not defined in terms of abstract or objective units of time such as 
seconds, hours, or years. The frames are qualitatively distinguished in 
terms of the different types of underlying processes and 
causal-conditional mechanisms that define them. For each frame, what 
matters most is how it works, not how long it takes. 



By offering a scheme of interrelated causal frames as a conceptual 
framework for research on language, I want to stress two points. 



The first is that these frames are most useful when we keep them 
conceptually distinct. Kinds of reasoning that apply within one frame do 
not necessarily apply in another, and data that are relevant in one 
frame might not be relevant (in the same ways) in another. We need to 
avoid the chronic confusions that arise from mixing up these frames. 



The second point is that in order to get a full understanding of the 
phenomena we study it is not just necessary to understand these 
phenomena from within all of the different frames. The ideal is also to 
show how each frame is linked to each other frame, and, ultimately, how 
together the frames point to the existence of a system of causal forces 
that define linguistic reality.



\section{Distinct frames and forces}


The ethologist Niko Tinbergen, in a 1963 paper \textquoteleft On aims and methods in 
Ethology', emphasized that different kinds of research question may be 
posed within different theoretical and methodological frames, and may 
draw on different kinds of data and reasoning. When studying animal 
behavior, Tinbergen argued, some questions may be answered with 
reference to the development of an individual organism, others with 
reference to the evolution of a species, others may concern the proximal 
mechanism of a pattern of behavior, and yet others may have to do with 
the survival or fitness value of the behavior, independent from the 
other three kinds of question. See Table 1:



\begin{table}[h]
\centering
\begin{tabular}{|l|l|}
\hline
\textbf{Causal} & What is the mechanism by which the behavior occurs?
\\
\hline
\textbf{Functional} & What is the survival or fitness value of the 
behavior? \\
\hline
\textbf{Phylogenetic} & How did the behavior emerge in the course of 
evolution? \\
\hline
\textbf{Ontogenetic} & How does the behavior emerge in an individual's 
lifetime? \\
\hline
\end{tabular}
\end{table}


Table 1. Four distinct causal/temporal frames for studying animal 
behavior (after Tinbergen 1963).



Tinbergen's four questions were applied in studying the behavior of 
non-human animals. These distinctions were designed to handle 
communication systems such as the mating behavior of stickleback fish, 
not the far greater complexities of language, nor the rich cultural 
contexts of language systems. If we are going to capture the spirit of 
Tinbergen's idea, we need a scheme that better covers the phenomena 
specific to language and its relation to human diversity. 



As it happens, many researchers of language and culture have emphasized 
the need to monitor and distinguish different causal frames (often 
called \textquoteleft timescales') that determine our perspective. These include 
researchers of early last century (Saussure 1959; Vygotsky 1962) through 
to many of today (Tomasello 2003; Macwhinney 2005; R\k{a}czaszek-Leonardi 
2010; Cole 2007; Donald 2007; Larsen-Freeman and Cameron 2008; Uryu, 
Steffensen, and Kramsch 2013; Lemke 2000, 2002). Let us consider some of 
the distinctions they have offered.



The classical two-way distinction made by Saussure (1959) between the 
\textsc{synchronic} study of language-viewing language as a static 
system of relations-and the \textsc{diachronic} study of 
language-looking at the historical processes of change that give rise to 
the synchronic relations observed-is the tip of the iceberg. Those who 
have looked at the dynamic nature of language have quickly noticed that 
diachrony-in the usual sense of the historical development and 
divergence of languages-is not the only dynamic frame. 



Vygotsky distinguished between \textsc{phylogenetic}, \textsc{
ontogenetic}, and \textsc{historical} processes, and stressed that 
these dynamic frames were distinct from each other yet interconnected. 
His insight has been echoed and developed in much subsequent work, from 
psychologists of communication like Cole (2007) and Tomasello (1999) to 
computational linguists like Steels (1998; 2003) and Smith, Brighton, 
and Kirby (2003), among many others. 



Smith et al (2003, 540) argue that to understand language we must see it 
as emerging out of the interaction of multiple complex adaptive systems, 
naming three \textquoteleft time-scales' that need to be taken into account-\textsc{
phylogenetic}, \textsc{ontogenetic}, and \textsc{glossogenetic} 
(= \textquoteleft cultural evolution', i.e., diachronic)-thus directly echoing 
Vygotsky. Language is, they write, \textquoteleft a consequence of the interaction 
between biological evolution, learning and cultural evolution' (Smith, 
Brighton, and Kirby 2003, 541). 



R\k{a}czaszek-Leonardi, arguing that multiple frames need to be addressed 
simultaneously in psycholinguistic research, invokes three frames-
\textsc{online}, \textsc{ontogenetic }and \textsc{diachronic}
-thus not invoking the phylogenetic frame, but adding the \textquoteleft online' frame 
of cognitive processing. 



Cole (1996, 185) expands the list of dynamic frames to include \textit{
microgenesis}, \textsc{ontogeny} (distinguishing early learning from 
overall lifespan), \textsc{cultural history}, \textsc{phylogeny}, 
and even \textsc{geological time}. 



And Macwhinney (2005, 193ff) offers a list of \textquoteleft seven markedly different 
time frames for emergent processes and structure', citing Tinbergen's 
mentor Konrad Lorenz (1958). MacWhinney's frames are \textsc{
phylogenetic}, \textsc{epigenetic}, \textsc{developmental}, 
\textsc{processing}, \textsc{social}, \textsc{interactional}, 
and \textsc{diachronic}. 



Newell (1990, 122) proposes a somewhat more mechanical division of time 
into distinct \textquoteleft bands of cognition' (each consisting of three \textquoteleft scales'), 
taking the abstract/objective temporal unit of the second as his key 
unit, and defining each timescale on a gradient from 10$^{-4}$ 
seconds at the fast end to 10$^{7}$ seconds at the slow end: the 
\textsc{biological band} (= 10$^{-4}$-10$^{-2}$ seconds), the 
\textsc{cognitive band} (= 10$^{-1}$-10$^{1}$ seconds), the 
\textsc{rational band} (= 10$^{2}$-10$^{4}$ seconds), and the 
\textsc{social band} (= 10$^{5}$-10$^{7}$ seconds). He also 
adds two \textquoteleft speculative higher bands': the \textsc{historical band} (= 
10$^{8}$-10$^{10}$ seconds), and the \textsc{evolutionary band} 
(= 10$^{11}$-10$^{13}$ seconds; 1990, 152), thus suggesting a 
total of 18 distinct timescales. 



Like Newell (though without reference to him), Lemke (2000: 277) takes 
the second as his unit and proposes no less than 24 \textquoteleft representative 
timescales', beginning with 10$^{-5}$ seconds-at which a typical 
process would be \textquoteleft chemical synthesis'-through to 10$^{18}$ 
seconds-the scale of \textquoteleft cosmological processes'. 



Lemke's discussion is full of insights, but the taxonomy is generated by 
an arbitrary carving-up of an abstract gradient rather than being 
established in terms of research-relevant qualitative distinctions or 
methodological utility, or being derived from a specific theory (cf. 
Uryu, Steffensen, and Kramsch in press; Larsen-Freeman and Cameron 2008, 
169). It is not clear why a distinction between units of 3.2 years 
versus 32 years should necessarily correlate with a distinction between 
processes like institutional planning versus identity change; nor why 
the process of evolutionary change should span three timescales (3.2 
million years, 32 million years and 317 million years) or why it should 
not apply at other timescales. 



Larsen-Freeman and Cameron (2008: 169) propose a set of \textquoteleft timescales 
relevant to face-to-face conversation between two people': a \textsc{
mental processing }timescale of milliseconds, a \textsc{microgenetic }
timescale of online talk, a \textsc{discourse event} timescale, a 
\textsc{series of connected discourse events}, an \textsc{
ontogenetic }scale of an individual's life, and a \textsc{phylogenetic
} timescale. Uryu et al critique this model for not explaining why these 
timescales are the salient or relevant ones, and for not specifying 
which other timescales are \textquoteleft real but irrelevant'. 



Uryu et al propose a principled \textquoteleft continuum' of timescales running from 
'fast' to \textquoteleft slow' (11 distinctions in the order \textsc{atomic}, 
\textsc{metabolic}, \textsc{emotional}, \textsc{autobiographical
}, \textsc{interbodily}, \textsc{microsocial}, \textsc{event}, 
\textsc{social systems}, \textsc{cultural}, \textsc{evolutionary
}, \textsc{galactic}) that are orthogonal to a set of \textquoteleft temporal 
ranges' running from \textquoteleft simple' to \textquoteleft complex' (six distinctions in the 
order \textsc{physical universe}, \textsc{organic life forms}, 
\textsc{human species}, \textsc{human phenotype}, \textsc{
dialogical system}, \textsc{awareness}). Uryu et al's approach 
applies the notion of ecology to the dynamics of language and its usage 
(see also Steffenson and Fill 2013).



Some of these schemes are well-motivated but incomplete: Saussure gives 
just one dynamic frame, leading us to wonder, for example, whether we 
should regard speech processing as nano-diachrony. Vygotsky, similarly, 
gives us three dynamic frames, but does not single out or 
sub-distinguish the \textquoteleft faster' frames of microgeny and enchrony; are we to 
think of these as pico-ontogeny? On the other hand, some schemes give us 
finer differentiation than we seem to need, or offer weak or arbitrary 
motivations for the distinctions made. What is needed is a middle way. 



\section{MOPEDS: A basic-level set of causal 
frames}


From the array of frames discussed in the previous section, defined in 
different ways and for different purposes, I'm going to suggest that six 
float to the top. I outline these below, and I suggest that they capture 
what is most useful about previous proposals. 



These six frames are relatively well understood, are known to be 
relevant to research, are well-grounded in prior work on language and 
culture, and are known to be related to each other in important ways. I 
suggest that this set is what we need: a basic-level set of conceptually 
distinct but interconnected causal frames for understanding language. 



Each of the six frames-microgenetic, ontogenetic, phylogenetic, 
enchronic, diachronic, synchronic-is distinct from the others in terms 
of the kind of causality it implies, and thus in its relevance to what 
we are asking about language and its relation to culture and other 
aspects of human diversity. One way to think about these distinct frames 
is that they are different sources of evidence for explaining the things 
that we want to understand. I now briefly define each of the six frames.



\subsection{\textit{Microgenetic (action production and perception/comprehension)}}



In a microgenetic frame, we consider the processes by which linguistic 
behaviours such as simple utterances are psychologically processed. For 
example, in the production of spoken utterances there is a set of 
cognitive processes-concept formulation, lemma retrieval, phonological 
encoding, etc.-that take us \textquoteleft from intention to articulation' (Levelt 
1989). Or in the perception and comprehension of others' utterances 
(Cutler 2012), there are cognitive processes by which we parse the 
speech stream, recognize distinct words and constructions, and infer 
others' communicative intentions. 



These processes tend to take place at time scales between a few 
milliseconds and a few seconds. The causal mechanisms involved at this 
level include working memory (Baddeley 1986), application of rational 
heuristics (Gigerenzer, Hertwig, and Pachur 2011), balancing of 
processing effort (Zipf 1949), online categorization, motor routines, 
inference, possession and attribution of mental states such as beliefs, 
desires, and intentions (Searle 1983; Enfield and Levinson 2006), and 
the fine timing of motoric control and action execution.



/subsection{\textit{Ontogenetic (biography)}}



In an ontogenetic frame, we are considering the processes by which an 
individual's linguistic capabilities and habits are acquired and/or 
change during the course of that individual's lifetime. Many of the 
phenomena that are studied within this frame come under the general 
rubric of language acquisition and socialization, referring primarily to 
the learning of a first language by infants (see Brown and Gaskins, 
2014; Clark 2009), but also referring to the learning of a second 
language by adults. 



The kinds of causal processes observed in this frame include strategies 
for learning, and motivations for learning. Some of these strategies and 
motivations can be complementary, and some may be employed at distinct 
phases of life. The causal processes involved in this frame include 
conditioning, statistical learning and associated mechanisms like 
entrenchment and pre-emption (Tomasello 2003), adaptive docility (Simon 
1990), a pedagogical stance (Gergely and Csibra 2006), and long-term 
memory.



\subsection{\textit{Phylogenetic (biological evolution)}}



A phylogenetic frame considers the processes by which our species came 
to acquire the capacity for learning and using language, including the 
cultural context of language. The study of the evolution of the language 
capacity in humans fits firmly within the broader field of study of the 
biological evolution and origin of species. It is a difficult topic to 
study, but this has not stopped a vibrant bunch of researchers from 
making progress (see Levinson, 2014). 



The kinds of causal processes that are at play within a phylogenetic 
frame are those typically described in evolutionary biology. They invoke 
concepts such as survival, fitness, and reproduction of biological 
organisms (Ridley 1997, 2004), which in the case of language means 
members of our species. 



The processes involved at this level include the basic elements of 
Darwinian natural selection: competition among individuals in a 
population, consequential variation in individual characteristics, 
heritability of those characteristics, and so forth (Darwin 1859; 
Dawkins 1976; Mayr 1982).



\subsection{\textit{Enchronic (interactional)}}



An enchronic frame views language in the context of the interactional 
sequences of moves, as Goffman termed them, that constitute typical 
communicative interaction. In an enchronic frame, the causal-conditional 
processes of interest involve both structural relations of sequence 
organization (practices of turn-taking and repair which organize our 
interactions; Schegloff 1968; Sacks, Schegloff, and Jefferson 1974; 
Schegloff, Jefferson, and Sacks 1977; Sidnell and Stivers 2012; Enfield 
and Sidnell 2014) and ritual or affiliational relations of 
appropriateness, effectiveness, and social accountability (Heritage 
1984; Atkinson and Heritage 1984; Stivers, Mondada, and Steensig 2011; 
Enfield 2013). 



The linguistic phenomenon of turn-taking operates in the enchronic 
frame, as do the range of speech act sequences such as question-answer, 
request-compliance, assessment-agreement, and suchlike (see Sidnell and 
Enfield, 2014). Enchronic processes tend to take place at a temporal 
granularity around the one-second mark; from fractions of seconds up to 
a few seconds and minutes (though as stressed here, time units are not 
the definitive measure; exchanges made using email or surface mail may 
be stretched out over much greater lengths of time). 



Enchronic processes and structures are the topic of research in 
conversation analysis and other traditions of research on communicative 
interaction. Some key causal elements in this frame include relevance 
(Garfinkel 1967; Grice 1975; Sperber and Wilson 1995), local motives 
(Schutz 1970; Leont'ev 1981; Heritage 1984), sign-interpretant relations 
(Kockelman 2005, 2013; Enfield 2013, chap. 4) and social accountability 
(Garfinkel 1967; Heritage 1984).



\subsection{\textit{Diachronic (social/cultural history)}}



In a diachronic frame, we look at elements of language as historically 
conventionalized patterns of knowledge and/or behaviour. If the question 
is why a certain linguistic structure is the way it is, a diachronic 
frame looks for answers in terms of processes that operate in historical 
communities. While of course language change has to be \textquoteleft actuated' at a 
micro level (Weinreich, Labov, and Herzog 1968; Milroy 1980; Labov 1986; 
Eckert 2000), for a linguistic item to be observed in a language, that 
item has to have been diffused and adopted throughout a population 
before it can have become conventionalized. 



Among the causal processes of interest in a diachronic frame are the 
adoption and diffusion of innovations, and the demographic ecology that 
supports cultural transmission (Rogers 2003). Population-level 
transmission is modulated by the microgenetic processes of conceptual 
extension, inference, and reanalysis that feed grammaticalization 
(Hopper and Traugott 1993). 



Of central importance in a diachronic frame are social processes of 
community fission and fusion (Aureli et al. 2008), migration (Manning 
2005), and sociopolitical relations through history (Smith 1776; Marx 
and Engels 1947; Runciman 2009). The timescales of interest in a 
diachronic frame are often stated in terms of years, decades, and 
centuries.



\subsection{\textit{Synchronic (representational)}}



Finally, a synchronic frame is distinct from the other frames mentioned 
so far because time is removed from consideration, or at least 
theoretically so. One might wonder if it is a causal frame at all. But 
if we think of a synchronic system as a veridical statement of the items 
and relations in a person's head, as coded, for example, in their 
memory, then this frame is real and relevant, with causal implications, 
even if we see it as an abstraction (e.g., as bracketing out 
near-invisible processes that take place in the fastest levels of 
Newell's \textquoteleft biological band'). 



In Saussure's famous metaphor, language is a game of chess. If we 
observe the state of the game at the halfway point, a diachronic account 
would describe what we see in terms of the moves that had been made up 
to that point, and that had created the situation we observe. A 
synchronic account would simply describe the positions and 
interrelations of the pieces on the board at that point in time. For an 
adequate synchronic description, one does not need to know how the set 
of relations came to be the way it is. 



There are two ways to take this. One is to see it as a purely 
methodological move, an abstraction that allows the professional 
linguist to describe a language as a whole system that hangs together. 
Another, which is not in conflict with the first, is to see the 
synchronic description of a language as a hypothesis about what is 
represented in the mind of somebody who knows the language. 



A synchronic system cannot be an entirely atemporal concept---at the very 
least because synchronic structures cannot be inferred without 
procedures that require time; e.g., the enchronic sequences that feature 
in linguistic elicitation with native speaker informants-but it is 
clearly distinct from a set of ontogenetic processes, on the one hand, 
and diachronic processes, on the other (though it is of course causally 
implied in both). We can infer an adult's knowledge of language and 
distinguish this from the processes of learning that led to this 
knowledge, and from the historical processes that created the model for 
this knowledge but which neither the learner nor the competent speaker 
need have had access to. 



Because the concern here is on characterizing different frames of 
relevance to a natural, causal account of language, when I talk about a 
synchronic frame I shall mean it as a hypothesis about an adult's 
conceptual representations of a language that makes it possible for them 
to produce and interpret utterances in the language. 



The causality in a synchronic frame is tied to the events that led to 
the knowledge, and the events that may lead from it, as well as how the 
nature and value of one convention may be dependent on the nature and 
value of other conventions that co-exist as elements of the same system 
(Smith 1776; Marx and Engels 1947; Runciman 2009).



\subsection{\textit{Interrelatedness of the frames?}}



A challenge that awaits us is to figure out how these frames are 
interrelated. As R\k{a}czaszek-Leonardi (2010, 276) says, \textquoteleft even if a
\textit{ }researcher aims to focus on a particular scale and system, he 
or she has to be aware of the fact that it is embedded in others'. Other 
authors (Cole 1996, 179; Macwhinney 2005, 192) have asked: What are the 
forces that cause these frames to \textquoteleft interanimate' or \textquoteleft mesh'? One place to 
begin would be to test and extend the suggestions of authors like Newell 
(1990), Cole (1996, 184-5), MacWhinney (2005), Lemke (2000, 279ff) and 
Uryu et al (2013). 



How might the outputs of processes foregrounded within any one of these 
explanatory frames serve as inputs for processes foregrounded within any 
of the others? The answers will greatly enrich our tools for 
explanation.



\section{The case of Zipf's length-frequency rule}


The main payoff of having a set of distinct causal frames for language 
is that it offers explanatory power. By way of illustration, let us 
consider a simple case study, the observation made by Zipf (1935; 1949), 
that \textquoteleft every language shows an inverse relationship between the lengths 
and frequencies of usage of its words' (1949, 66).\footnote{I would like to thank Martin Haspelmath for insisting on the distinction between Zipf's law and Zipf's length-frequency rule. Zipf's Law states that there is a correlation between the frequency of an item and its frequency rank relative to other items in a set.}



Zipf suggested that the correlation between word length and frequency is 
ultimately explained by a psychological preference for minimizing 
effort. If we take that as a claim that synchronic structures in 
language are caused by something psychological-though Zipf's own claims 
were rather more nuanced-this raises a \textquoteleft linkage' problem (Enfield 2002, 
18, citing Clark and Malt 1984, 201). 



The problem is that a cognitive bias or preference, such as a desire to 
minimize effort, cannot directly affect a synchronic system's structure. 
A cognitive preference is a property of an individual while a synchronic 
fact is shared across an entire population. Something must link the two. 
While it may be a fact that the relative length of the words I know 
correlates with the relative frequency of those words, this fact was 
already true of my language (i.e., English) before I was born. The fact 
of a correlation cannot, therefore, have been caused by my cognitive 
preferences. How, then, can the idea be explicated in causal terms?



As was clear to Zipf (1949, 66ff), to solve this problem we appeal to 
multiple causal frames. We can begin by bringing diachronic processes 
into our reasoning. The presumption behind an account like Zipf's is 
that all members of a population have the same biases. 



The key to understanding the link between a microgenetic bias like 
'minimize effort in processing where possible' is to realize that this 
cognitive tendency has an effect only in its role as a \textit{
transmission bias} in a diachronic process of diffusion of convention 
in a historical population (see below chapters for explication of 
diachrony as an epidemiological process of biased transmission, 
following Rogers 2003; Sperber 1985; Boyd and Richerson 1985; Boyd and 
Richerson 2005, among others). 



The observed synchronic facts are an aggregate outcome of the individual 
biases multiplied in a population and through time. The bias has a 
causal effect precisely in so far as it affects the likelihood that the 
pattern will spread throughout a population. 



Now, while the spread of a pattern and its maintenance as a convention 
in a population are diachronic processes, the operation of a 
transmission bias may occur in three other frames. 



In an ontogenetic frame, a correlation between the shortness of words 
and the frequency of words might make the system easier to learn, and 
might thereby introduce a bias that causes the correlation to become 
more widely distributed in a population. 



In a microgenetic frame, individuals may be motivated to save energy by 
shortening a word that they pronounce often, again broadening the 
distribution of the correlation in a population. 



And an enchronic frame will capture the fact that communicative behavior 
is not only regimented by biases in learning or individual-centred 
preferences of processing and action, but also by the need to be 
successfully understood by another person if one's communicative action 
is going to have the desired effect. It is the presence of an 
interlocutor who displays their understanding, or failure thereof, in a 
next move-criterial to the enchronic frame-that provides a selectional 
counter-pressure against the tendency to minimize effort in 
communicative behavior. There is a need to have one's action recognized 
by another person if that action is going to be consummated (Zipf 1949, 
21; Enfield 2013, chap. 8). 



By adopting a rich notion of a diachronic frame in which transmission 
biases play a central causal role, we can incorporate the ontogenetic, 
microgenetic and enchronic frames in explaining synchronic facts, 
invoking the mechanisms of \textit{guided variation} explicated by 
Boyd and Richerson (1985; 2005) and explored in subsequent work by 
others (e.g., Kirby 1999; Kirby, Smith, and Brighton 2004; Christiansen 
and Chater 2008; Chater and Christiansen 2010). This allows us to hold 
onto Zipf's insight, along with similar claims by other authors such as 
Sapir before him, and Greenberg after him, who both also saw connections 
between psychological biases and synchronic facts. 



Greenberg (1966) implied, for example, that there is a kind of cognitive 
harmony in having analogous structures in different parts of a language 
system. Sapir (1921, 154-158) suggested that change in linguistic 
systems by drift can cause imbalances that produce \textquoteleft psychological 
shakiness', leading to the need for grammatical reorganization to avoid 
that mental discomfort. 



Similar ideas can be found in work on grammaticalization (cf. Giv\'{o}n 
1984; Bybee 2010) and language change due to social contact (cf. 
Weinreich 1953), supporting the notion that synchronic patterns can have 
psychological explanations but only when mediated by the aggregating 
force of diachronic processes. 



The point is central to explaining a range of other observed 
correlations, for example that more frequent words change more slowly 
(Pagel, Atkinson, and Meade 2007), that differences in processes of 
attention and reasoning correlate with differences in the grammar of the 
language one speaks (Whorf 1956; Lucy 1992; Slobin 1996, inter alia), 
that ways of responding in conversation can be constrained by collateral 
effects of language-specific grammatical structures (Sidnell and Enfield 
2012), and that different cultural values can give rise to different 
grammatical categories (Hale 1986; Wierzbicka 1992; Enfield 2002 and 
references therein). 



Most if not all of these claims bracket out some elements of the full 
causal chain involved. To give a complete and explicit account of the 
causal chain involved, a multi-frame account is needed.



\section{?}


\textquoteleft We might gain considerable insight into the mainsprings of human 
behavior', wrote Zipf (1949, v), \textquoteleft if we viewed it purely as a natural 
phenomenon like everything else in the universe'. This does not mean 
that we cannot embrace the anthropocentrism, subjectivity and 
self-reflexivity of human affairs. It does mean that underneath all of 
that, our analyses remain accountable to natural, causal claims. 



If we are going to answer the two fundamental questions \textquoteleft What's language 
like?' and \textquoteleft Why is it like that?' we will have to look in multiple 
directions for answers. Our case study of the length/frequency 
correlation in words invoked cognitive preferences of individual agents 
and related these to formal features of community-wide systems. To 
establish links between these, we drew on a set of frames---under the 
rubric of MOPEDS---that are distinguished from each other in terms 
of different types of causal process and conditional structure, and that 
roughly correlate with different timescales. These different types of 
process together account for how meaning arises in the moment. \newpage