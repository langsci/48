\chapter{Conclusion}

Ever since Darwin's earliest remarks on the uncanny similarity between 
language change and natural history in biology, there has been a 
persistent conceptual unclarity in evolutionary approaches to cultural 
change. This unclarity concerns the units of analysis. 



In some cases the unit is said to be the language system as a whole. A 
language, then, is \textquoteleft like a species' (Darwin 1871, 60; cf. Mufwene 2001, 
192-194). One reading of this is that we are working with a population 
of idiolects that is coterminous with a population of bodies (allowing, 
of course, that in the typical situation---multilingualism---one body houses 
two or more linguistic systems). 



On another view, the unit of analysis is any unit that forms \textit{part} of a language, such as a word or a piece of grammar. \textquoteleft A struggle 
for life is constantly going on amongst the words and grammatical forms 
in each language' (Max M\"{u}ller 1870, cited in Darwin 1871, 60). In 
contrast with the idea of populations of idiolects, this suggests that 
there are populations \textit{of items }(akin to Zipf's economy of 
word-tools), where these items are reproduced, and observed, in the 
context of spoken utterances. 



While some of us instinctively think first in terms of items, and others 
of us first in terms of systems, we do not have the luxury of ignoring either. 
Neither items nor systems can exist without the other, and the challenge 
is to characterize the relation between the two---this relation being the 
one thing that defines them both. 



The issue is not just the relative status of items and systems but the 
causal relations between them. If the distinction 
between item and system is a matter of framing, it is no less 
consequential for that. We not only have to define the differences 
between item phenomena and system phenomena, we must know which ones we 
are talking about and when, and we must show whether, and if so how, we 
can translate statements about one into statements about the other. 



In this chapter we have adopted a causally explicit model for the 
transmission of cultural items, and we have approached a solution to the 
item/system problem that builds solely on these item-based biases. I 
submit that the biases required for item evolution---never forgetting that 
\textquoteleft item' here really means 
\textquoteleft something-and-its-functional-relation-to-a-context'---are sufficient not 
only to account for how and why certain cultural items win or lose, they 
also account for the key relational forces that \textit{link }items 
and systems. 



In the last chapter, we started with the puzzle of the item/system problem. To solve it, we 
reached for the most tangible causal mechanism we have for the existence 
of linguistic and cultural reality: item-based transmission. The idea I 
have tried to put forward here is that by defining items more 
accurately---as always having a functional relation to context---we can have 
an item-based account for linguistic and cultural reality that gives us 
a system ontology for free.

\section{Conclusion notes}


Consider the idea of a whole \textit{grammar}, not one sign but an enormous complex of interrelated signs, with syntagmatic, paradigmatic, and hierarchical relationships implicit, the emphasis of individual embodiment must remain. The only model with any hope of empirical validity is that of the \textit{idiolect}. Each individual carries all their personally constructed sign-meaning hypotheses and sign associations with them, as they move around during the day, drawing on this in linguistic action all the time. And these idiolects overlap to greater or lesser degrees. Entire linguistic systems, as empirically definable ‘objects’, do exist but are unique to each individual. A ‘grammar’ is a massive object under the centripetal force of daily, socially situated usage, in which the hypotheses of one’s idiolect are ceaselessly tested, and usually confirmed and further entrenched. Other notions of grammatical system (i.e. ‘langue’, grammar floating in space) are only ideas in themselves, abstractions, in the minds of those who talk about ‘languages’, contrasting ‘the way \textit{we} speak’ from ‘the way \textit{they} speak’. This is where ideas like ‘the English language’ emerge. Linguists’ descriptive grammars are extrapolations based on idiolects (a fact which does not detract from their great value), and the system described is an imaginary one, the ‘average’ of those idiolects which provided the evidence. The description cannot claim to represent something real, unless it is based on data from a single speaker.

Consider in the present context what it means to say that there are \textit{rules }of grammar, if they are not imposed upon us from outside of our selves? Le Page and Tabouret-Keller have described linguistic ‘rules’ as probabilistic (Bickel REF), extrapolated from observed regularity: ‘Most linguistic descriptions have in fact been based on the observation of regularities in behaviour, but the rules derived in this way tend to become idealized and converted into a more firmly predictive kind of rule.’ (1985:194), and ‘every use of language is a fresh application, a metaphorical extension, of existing systems, made at risk’ on the basis of these probabilistic ‘rules’. (1985:196) Jesperson described it 90 years ago:

\begin{quotation}
	My chief object... has been to make the reader realise that language is not exactly what a one-sided occupation with dictionaries and the usual grammars might lead us to think, but a set of habits, of habitual actions, and that each word and each sentence spoken is a complex action on the part of the speaker. The greater part of these actions are determined by what he has done previously in similar situations, and that again was determined chiefly by what he had habitually heard from others. But in each individual instance, apart from mere formulas, the speaker has to turn these habits to account to meet a new situation, to express what has not been expressed previously in every minute detail; therefore he cannot be a mere slave to habits, but has to vary them to suit varying needs---and this in course of time may lead to new turns and new habits; in other words, to new grammatical forms and usages. Grammar thus becomes a part of linguistic psychology or psychological linguistics...  (Jesperson 1924:29)
\end{quotation}



	While I have argued that grammar resides in the individual, it may now be added that the grammars of individuals in social association are highly convergent. As Strauss and Quinn (1997:16) put it, 'It is a fallacy to assume that if meanings are intrapersonal (private in the sense of not being observable by others), they are therefore idiosyncratic (private in the sense of not being held in common).' To the contrary, there is great overlap (though never perfect) of the structured constellations of linguistic categories called idiolects, and this overlap is achieved through the \textit{centripetal} effects of mutual/communal social saturation. People in constant contact fall naturally into patterns of \textit{convention }and \textit{consensus} regarding their use of linguistic categories. The sheer volume of linguistic categories we introduce into each other's environment forces us into a collective \textit{holding pattern }with regard to our respective hypotheses of linguistic sign meaning. While our hypotheses of sign meanings \textit{are} revisable and negotiable, revision or negotiation less often take place, since we have neither the time nor (usually) the need to challenge convention. Negotiation of sign meaning (beyond what \textit{has already been established} and is \textit{now inter-recursively assumed}) is costly and preferably avoided. 


In special social circumstances such as those surrounding the emergence of creoles and pidgins, variation in the idiolects of individuals in constant social association may display great variation, and relatively few commonly established linguistic signs may be reliably employed. Social factors then apply centripetal force in encouraging a process of language \textit{focus}, a term Le Page and Tabouret-Keller use to describe the process of idiolects converging and becoming more concretised in socially directed ways:

We can recognise among the agencies which promote focusing: (a) close daily interaction in the community; (b) an external threat or any other danger which leads to a sense of common cause; (c) a powerful model---a leader, a poet, a prestige group, a set of religious scriptures; (d) the mechanisms of an education system. (Le Page and Tabouret-Keller 1985:187) All of these can be understood as biases on transmission.

‘Public’ status of ‘the system’ is emergent, epiphenomenal - Consistencies among idiolects, the fact of mutual intelligibility, and the system properties of language which are so well documented, are not incompatible with this individual-based view. Idiolect convergence arises at least as the result of \textit{focussed social association} (along with possible further focus from standardised descriptions of the data). The ‘system’ is public due to a \textit{centripetal} feedback process arising from (1) a huge number of signs in constant usage; (2) small numbers of people in convergent focussed/high-intensity interaction; (3) lack of time in the day to negotiate or challenge or invent sign meanings - the easiest thing to do is to fall back on \textit{convention} (especially given the ‘articulation bottleneck’).

Language and social group identification
We are naturally inclined to assume \textit{identity }in focused social groups, and we hammer each other with language within those groups, such that our level of communication is very high (biologically good) as well as our level of focus being very high (biologically good; cf. Dunbar/Nettle on trust; Jane Hill on different levels of focus in different situations - a very immediate way to identify \textquoteleft one of us' is by how they speak, and the distinctions may be at a large range of levels; cf. two different languages versus two different gangs at school). There is a ‘natural’ stop-and-chat social group size of about 150 people (Dunbar Gladwell 2000), which in terms of hunter-gatherer societies would include not just one’s own band, but the larger group one may regularly find oneself in at annual events. (Grooming)
	Why do we speak the way we do? Because that's how we speak:

	Anyone who has ever asked unsophisticated Chinese informants why they follow such and such a custom knows the maddeningly reiterated answer: \textquoteleft Because we are Chinese'. At first one assumes that this is simply a stock response to the uncultured foreigner or a way of fobbing off an impertinent outsider; after a time one realizes that most of one's informants do themselves see it as the correct explanation of almost all their own cultural behaviour and social organisation. (Ward 1965:113-114.)

	This is an important principle on both conscious and unconscious levels. It operates most fundamentally with a sense of ‘people’ (of whom I am one), and also often in a negative sense (i.e. if people don’t have a sense of ‘how we say it’, they certainly will have a sense of ‘how we \textit{don’t} say it’, or perhaps ‘how \textit{they} say it’).  This is where the findings of sociolinguistics must be directly admitted into the analysis of linguistic diffusion. 

Linguistic encounters involve great risk to face, are emotionally loaded, since we are manipulating \textit{identity }all the time.) LePage (1968:192) writes:

	Each individual creates the systems for his verbal behaviour so that they shall resemble those of the group or groups with which from time to time he may wish to be identified, to the extent that
(a) he can identify the groups,
(b) he has the opportunity and ability to observe and analyse their behavioural systems,
(c) his motivation is sufficiently strong to impel him to choose, and to adapt his behaviour accordingly,
(d) he is still able to adapt his behaviour.


Naturally, social processes are crucial. But psychology is as important to language as has long been supposed, but that the view of psychology assumed in most linguistics needs to be updated.


I want to put forward a theoretical position on language, as a sociocultural phenomenon, which borrows from a host of areas in current interdisciplinary research, especially from ethology and related work. The position brings up some fundamental questions about semiotic ‘systems’. The first aspect of this position is that the atomic unit of analysis is the individual ‘item’, and that the ‘system’ - in two senses - emerges from this and is secondary to this. In society, the individual is physically detached from his associates, and carries at each moment a set of social capacities which are transportable ‘in his skin’. At the same time, however, he exercises those capacities in direct association with others, his capacities are continually practised on others, reliant upon co-operative interaction with others, and thus highly convergent with those others. A genuine social ‘system’ beyond the individual emerges from this convergence. And a second sense of social ‘whole’ coinciding with this to some extent is the conscious, attributed \textit{notion of} a whole, which may emerge as an ethnonym or village/town name - i.e. an identified ‘society’ or ‘culture’. Similarly, in language, the individual linguistic expression - word, morpheme, idiomatic phrase, identifiable linguistic practice - is separable and detachable, and it is not \textit{necessarily} transmitted as part of any system. However, there is a very high level of co-occurence of signs among a particular large set, in given social situations. We are capable of handling many thousands of linguistic signs, given our huge capacity for storage in memory and very fast symbol processing speed. But the potential proliferation in variety and number of these signs is bottlenecked by important restrictions, which ultimately reduce to \textit{time} \textit{constraints}. First, we are constrained by our slow rate of speech \textit{production}, which is very much slower than our processing speed (Levinson 1995). We can only say or hear or read so many words in a day. Second, we are constrained by the limited time we have for interaction with individuals in our social circles. 

Linguistic systems (i.e. grammars) have their existential locus in individuals, whose average biography lasts well under a hundred years, and that ‘a language’ beyond this is an idea in the minds of speakers who identify it, not a real thing. We must now ask what this implies for the familiar idea of the historical continuity of ‘a language’ over centuries. The history of ‘a language’, of the life of an organism-like grammatical system independent of the life of any one human carrying that system (as an idiolect), is purely illusory, a product of the ‘language’ as ‘thing’ metaphor (Durie and Ross 1996). There is no genuine continuity in linguistic or cultural history, especially when no symbolic mediating structure (e.g. written records) endures. Even when we do have writing, whose life span can transcend generations, recall that signs themselves, such as ink arrangements on a page, mean nothing without a living person to attribute meaning to them. The particular system which enables that meaning-attribution endures only as long as the life of the person embodying it. So the normal idea of the history of ‘a language’ is, like the very idea of ‘the language’, an idea in itself, and not an empirically real thing. There is no ‘language’ enduring through generations. There is only successive and overlapping individual reproductions of similar and often effectively identical highly complex patternings of linguistic signs.
\\
Thomason and Kaufman’s idea of ‘normal transmission’ is in danger of conveying the impression that language learners are not normally exposed to linguistic habits of people other than their parents and others of the same group. This view that linguistic communities, like cultural groups, are naturally or normally or properly sealed off from other groups in their own little equilibrium is one that anthropology has long discarded (cf. Layton 1997, Leach 1982), due especially to its not being supported by the facts. Leach in particular has argued this point strongly (x-ref, above). 
It is the rule for human groups to be in contact with other human groups (i.e. those nearby who are ‘not our people’), and this means that these groups have social network ties, are playing out real-time face-threatening engagements with those people, and most importantly in this context are exposing each other to ideas for ways of saying things. When someone gets an idea for a way of saying something from outside their own group, and adopts this in their own linguistic behaviour, let us call this ‘outside transmission’. In terms of what actually goes on in transmission, the \textit{mechanisms }of normal versus outside transmission are \textit{exactly the same}. The learner is exposed to the idea, and then, content to be identified as someone who would say that, says it. The difference is that when you get the idea from one of your own people, this identification as one who would say it, is effortless. You already are one of those people. When the idea comes from outside, and you adopt it, your willingness to be identified as one who would say that is either identification with that outside group, or with some other (new) social identity, presumably a sub-identity \textit{within }your own group. One crucial matter within normal transmission is that step two (consent to adopt the behaviour, knowing that you will be seen as ‘someone who adopts that behaviour’) is the default option, because you already are one. The difference is that a new behaviour picked up through outside source is associated with ‘other people’ are adopted.
\\
\textit{on internal development:}
Adults catch new ideas for ways of saying things from one of two sources; (1) from their own internal resources (i.e. they make up the new idea themselves), or (2) from some exposure source in their network.
			
New ideas from internal resources

	The study of how cognitive tendencies, or natural cognitive connections account for semantic relationships is essentially concerned with looking at new ideas for ways of saying things which lots of people are likely to independently think up themselves. A typical example is the use of body-part terms to describe spatial orientation (e.g. ‘back’). 
	There are some extensions which are more idiosyncratic and are not found across the world’s languages. One example is the pan-Southeast Asian description of dental caries as having ‘insects in the teeth’. These are ideas which people would be less likely to think up independently, but which, once one is exposed to them, make perfect sense. (These are akin to cultural practices like transporting liquid food and drinks in plastic bags, or putting sand shoes in the washing machine.)

New ideas from external sources (acquired, imported)

	We have been through the various exposure sources in one’s exposure network, including all the people one deals with as well as other sources of exposure to linguistic behaviour. 

A pattern like “all men” being bilingual would spread the innovation very effectively since families would all potentially be exposed. Perhaps if, say, five families became bilingual, an innovation may simply remain as a ‘private joke’ (in a technical sense), and never spread. 


	Apropos of Jakobson’s 1938 claim that a language ‘accepts foreign structural developments only when they correspond to its tendencies of development’ (Harris and Campbell 1995:123), Harris and Campbell remark that ‘in principle it ought to be easier to borrow constructions that are similar to existing ones (or at least do not conflict with the borrowing language basic structure) than structures that go against the structural grain of the borrowing language’ (Harris and Campbell 1995:124). Then, Harris and Campbell provide a number of counterexamples to this claim, stressing that it cannot be taken as absolute.

External and internal forces ‘compete’ in changing/determining/maintaining grammar. An external force may impact upon a language regardless of, or in spite of, or in contrast to, an internal tendency (a metaphor always?). Similarly, an internal tendency may resist/preclude an external force - say, the grammatical profile of a language precluding or making inapplicable a certain possible innovation. (For this, consider the relation in right-headed languages between subordinating multi-verb constructions [e.g. go-want, read-like] and iconically ordered multi-verb constructions [e.g. grab-get, fall-die].)

New signs must be made intelligible by seating them in familiar contexts; whole sign contexts are launching pads for new signs to make their premiere - One thing at a time...: Brilliantly stated by Herskovits:

The political revolutionary does not refuse to cast his revolutionary songs in the modal structure and scale progressions of the culture he is in the process of changing; his formations, if his organised forces are strong enough, will operate in terms of accepted patterns of military procedure. The one who rebels against th religious and moral system of his time will couch his appeals in the linguistic patterns of his people, use established affect symbols, and employ accepted esthetic standards in heightening the responses of his followers. (Herskovits 1951:153, quoted in Strauss and Quinn 1997:263)
\\
Geertz’ claims re: public symbolism took many interpretations. (S\& Q:13ff)
‘(G)roups do sometimes negotiate, exploit, reinterpret, borrow, and create cultural forms.... However, that they do so does not rule out internalised schemas. Even when intent on reinventing themselves, people do not pluck new cultural forms out of the air; their imaginings and reinterpretations always rely on understandings learned and imbued with motivation. Culturally variable, internalised schemas shape both the ways people define what is in their self-interest and the means they use to obtain those goals. This does not assume bounded cultural systems; learning can take place across national or ethnic borders as well as within them.’ (S\&Q:25)
(here, cf. Leach)


\textit{Five centripetal tendencies of cultural understandings}.
'First, they can be relatively durable in individuals. Secondly, cultural understandings can have emotional and motivational force, prompting those who hold them to act upon them. Thirdly, they can be relatively durable historically, being reproduced from generation to generation. Fourthly, they can be relatively thematic, in the sense that certain understandings may be repeatedly applied in a wide variety of contexts. Finally, they can be more or less widely shared.' (S\&Q:85)
\\
'The \textquoteleft rules' of a child's \textquoteleft native language' in this first sense are in any case likely to be tentative hypotheses, easily modified by fresh semantic needs, fresh contacts, fresh analogies. \textquoteleft Syntax' in the grammarian;s sense is what emerges from this process, not what it starts from.' (Le Page and Tabouret-Keller:190) cf. Hopper 1987
\textit{
}'Logic and universal grammar, then, are targets towards which, rather than the starting point from which, human linguistic activity proceeds. The origins of that activity are like those of a game which gradually develops among players, each of whom can experiment with changes of the rules, all of whom are umpires judging whether new rules are acceptable.' (Le Page and Tabouret-Keller:197)
\textit{+
}The notions of convention and agreement are crucial to my model. Cf, Chapter 5 of \textit{Meaning }by Stephen R. Schiffer. Oxford: Clarendon Press. 1972. Cf. also Lewis’ book \textit{Convention }(and Grice, and Schelling).
+
Le Page and Tabouret-Keller point to \textquoteleft the creativity of speakers, not just in the Chomskyan sense of using a finite set of rules to make up an infinite number of sentences, but in the sense of being able to make up the rules as one goes along. In the contact situations in which pidgin languages develop [and, in contact situations, generally] any ploy is tried which may make for communication under difficult circumstances - language is, as Roger Brown (1958) once called it, a game, in which the players invent the rules and, we would add, also act as umpires.' (Le Page and Tabouret-Keller:11) [Cf. Harris \& Campbell 1995:54]
\textit{+
}
Saussure - \textit{langue} is “a self-contained whole”, with a “natural order”.  Linguistic signs which constitute \textit{langue} are “associations which bear the stamp of collective approval” (p15).  \textit{Langue }“exists only by virtue of a sort of contract signed by the members of the community” (p14). \textit{Langue} is, in this sense, the “social side of speech”. (p 14) (cf Frege on \textquoteleft sense'.)

Crucial section the NCL book is on agency and the logic of communication; words as tool (CA, Zipf).

\textit{Some agency notes}
Collateral effects affect agency (+/-)

In all communication, the addressee is being played; our attention is a resource, and our inference is a tool and a resource for causing us to act (our behavior is a resource). Take the example of the evolution of red fruits for birds.



Zipf's effects are wholly concerned with the paradigmatic axis in linguistic systems, where alternatives compete for selection in the same slot. While such effects can be seen in the design and accessibility of tools in an artisan's workshop, we must note that paradigmatic effects acquire another level of systematization when the \textquoteleft tools' at issue have their effects by means of the \textit{recognition }of their function by an addressee (cf. Enfield 2013a: ) are linguistic signs is another I should also add that each serving as functional alternatives to the others, not only do they pick up extra meanings from this opposition (in the sense sketched above with reference to Darwin and Saussure).

Zipf 1949: ‘words are tools that are used to convey meanings in order to achieve objectives’ (p20)

; and grammatical systematization through content biases that relate to (a) psychological representation by individuals of large inventories of linguistic data and (b) functional motivations such as ease in production and recognizability of meaning in comprehension (e.g., as facilitated by iconicity and other sources of natural meaning)


