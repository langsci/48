%title = Ontology of Language Systems: Natural Causes of Language
\chapter{Language Transmission}

\begin{quotation}
\textit{Everything is this way because it got this way.}
D'Arcy Thompson
\end{quotation}



\section{The problem}
What is the causal relationship between linguistic items---individual sounds, words, idioms---and the whole systems of such items that we readily refer to as \textquoteleft languages'? A way into this question is to consider the simple problem of trying to explain why any two languages share a trait. There are four possible reasons (Enfield 2003:368):


\begin{list}{}{}
\item {0. \textsc{universal}: All languages must have the trait; therefore A and B have it.}

\item {1. \textsc{vertical transmission}:The trait was inherited from a common ancestor language; the ancestor language with the trait is a common ancestor to A and B.} 

\item {2. \textsc{internal development}: The trait was internally innovated by both A and B, independent from each other.}\footnote{If the two languages possessed the same starting conditions for the same internal innovation, the question arises as to why they shared those starting conditions in the first place.}

\item {3. \textsc{horizontal transmission}: The trait was borrowed into one or both of the languages (from A into B, from B into A, or from a third language into both A and B). }

\end{list}

%Anyone who has grappled with the problem of language contact and its historical effects will know the conceptual problems that arise from this question. Consider some of the many examples from around the world of neighbouring but unrelated languages that show many traits in common:

%\begin{list}{}{}
%\item 1.	In Northern California, native American languages of different stocks share features including phonetic/phonological systems and processes, MORE (REF);
%\item 2.	In Mesoamerica, languages of different stocks share features including head-marking possessive constructions, relational nouns, and vigesimal numeral systems (REF);
%\item 3.	In South Asia, languages of different stocks share features including retroflex consonants, conjunctive participles, (inter alia) among different stocks in .
%\item 4.	In the Balkans, languages of different stocks share features including post-posed articles, loss of infinitive in complement clauses, vowel harmony (inter alia).
%\item 5. 	In mainland Southeast Asia, languages of different stocks share features including (Enfield and Comrie 2015:7-8; cf. Enfield 2003, 2005);
%\end{list}


%The phenomena of language contact is usually discussed in terms of the properties of languages---what features languages have, how languages affect each other, what happens to languages. But if the empirical validity of the unit \textquoteleft language' as a causal entity is not a given, then we must wonder whether items are all we have.

The three possibilities (1-3) involve processes which are generally considered to be qualitatively distinct from each other, namely (1) inheritance (from mother to daughter language), (2) borrowing (from neighbouring language to neighbouring language through contact among speakers), and (3) natural, internally-motivated semantic development. But at a fundamental level they are not distinct. This is the conclusion I came to when assessing the implications of findings from research on the outcomes of contact between neighboring language communities in mainland Southeast Asia, and the linguistic change that results (Enfield 2005:198):

\begin{quotation}
All language change, whether by genealogical inheritance or areal diffusion, is conducted by a process of social diffusion of innovation. Once this is acknowledged, the analytical distinction between inheritance and diffusion begins to crumble. %Nevertheless, the genealogical method remains a useful descriptive technique.
\end{quotation}

\begin{quotation}
Areal linguistics invites us to revise our understanding of the ontology of languages and their historical evolution, showing that the only units one needs to posit as playing a causal role are individual speakers and individual linguistic items. These unit types are mobile or detachable with respect to the populations they inhabit, arguing against essentialism in both linguistic and sociocultural systems.
\end{quotation}


\begin{quotation}
Areal linguistics presents significant challenges for standard understandings of the ontology of language from both spatial and temporal perspectives. Scholars of language need to work through the implications of the view that \textquoteleft the language' and \textquoteleft the community' are incoherent as units of analysis for causal processes in the historical and areal trajectories of language diffusion and change.
\end{quotation}

Earlier (Enfield 2003: 368), regarding the oft-made distinction between inheritance of the same trait from a common ancestor, borrowing of a trait across languages, and parallel but independent internal development as accounts for similarity between languages, I wrote: 


\begin{quotation}
At a fundamental level, these three channels of a sign's entry into a language [i.e., inheritance, borrowing, internal development] are indistinguishable.
\end{quotation}


Then (Enfield 2005:197): 

\begin{quotation}
Language change by contact or otherwise is a process of social diffusion. The standard analytical distinction between internal and external linguistic mechanisms diverts attention from the fact that these are instances of the same process: the diffusion of cultural innovation in human populations.

\end{quotation}

The aim of this book is to consider what these conclusions mean, and to explicate the conceptual foundations of the problems of inheritance and diffusion in the history of languages. To solve these problems, we have to confront the item/system question posed at the beginning of this chapter. These considerations take us to one of the conceptual foundations of language science, the one that I want to discuss in this book. It is a question of causality: What causes a language to be the way it is, rather than some other way?
 
%The three processes mentioned above---inheritance, borrowing, language-internal innovation---can only take place when there is social contact between individuals, and successful diffusion of types of behaviour in populations. These are causal preconditions. For any of the three processes to take place, several things have to happen. People have to start saying things in a new way (or indeed saying new things), exposing those in their personal network to the idea. Those who are exposed then have to copy this new behavior (assuming they are motivated to do so). This in turn has to expose more people in their social networks, as well as further exposing those who began the process in the first place, validating and encouraging the usage, and leading it to take further hold. At a fundamental level, these three channels of a sign's entry into a language are indistinguishable from one another. If there are differences, they concern the source of the idea, the naturalness of the idea (i.e. how much it makes sense and perhaps how much it helps cut communicative and/or processing corners), and the social identificational value of the idea.


\section{Units of analysis in language history}
One way to understand a system is to look at the history of events that caused it to be the way it is. In the history of life forms, the set of formative events take place in populations, where individuals inherit characteristics---for example, from the genome of their parents. When  inherited characteristics vary between individuals in a group, it can mean that an individual with a certain variant has a better chance of surviving than someone with another variant. This, in turn, can increase the frequency of the advantageous variant in the population, and in time, the variant is carried by all individuals. This is how a population can diversify, splitting into two or more new populations, which can then be regarded as separate species. While they share a recent common ancestor, they are distinct groups. 

This way of thinking about the causal basis of species in terms of population dynamics is, of course, fundamental in the conceptual framework of biological evolution (Darwin REF, Mayr REF). It can be applied to the evolution of life forms of all kinds, and to cultural forms including kinship systems, technology, and language (REFS). Whatever forms of life we are considering, relations of common ancestry may be represented using a tree diagram. Figure 1 illustrates:

%\begin{figure}[h!]
%\includegraphics[width=0.90\textwidth,keepaspectratio]{figures/XXXX}
%\caption{Figure 1. Tree diagram representing divergence by descent with modification. A1, A2i, and A2ii are common descendants of A.})}
%\end{figure}

Diversification of languages, as viewed in research on the history of great language families like Bantu, Austronesian, or Indo-European, has long been represented with tree diagrams of this kind, in which the units of analysis are \textquoteleft languages'. The tree diagram is a methodological simplification that requires us to abstract from the facts. It may be that this abstraction is a harmless practical necessity. But our question is whether the abstraction inherent in the tree diagram does \textit{conceptual} harm. I will argue that it directs our attention away from the causal mechanisms that define language as an evolutionary process, and languages as evolved systems. 

The genetic metaphor in linguistics, by which languages and organisms are analogous, is generally considered to be harmless:

\begin{quotation}
[A] claim of genetic relationship [between a parent and a daughter language] entails systematic correspondences in all parts of the language because that is what results from normal transmission: what is transmitted is an entire language---that is, a complex set of interrelated lexical, phonological, morphosyntactic, and semantic structures. (Thomason and Kaufman 1988:11)
\end{quotation}

But the term \textquoteleft genetic relationship' is a figure of speech. What is transmitted between generations? It is not the whole system that, but the components of the system, piece by piece and chunk by chunk, in millions of distinct events. Not all at once but at separate moments, over days, weeks, months and years. The result is a degree of overlap among idiolects in a population that is so high that the idiolects are practically indistinguishable. How does this degree of idiolect overlap come about? The answer is that speech communities are inward-focussed. It is more or less the same small group of individuals who transmit individual items amongst each other. This is facilitated by any existing degree to which items are shared, and this inward focus in turn increases the degree of sharedness. This feedback effect in the social circulation of linguistic items is both an effect of, and a cause of, the common ground among individuals in a community. People interact more because they have greater common ground; they have greater common ground because they interact more. \footnote{On idiolect overlap or convergence, cf. Bakhtin (1981), Hockett (1987:106-7, 157-8), Lee (1996:227-8).}


\section{Horizontal versus vertical transmission}
The tree diagram assumes that languages cohere as systems. But is it established that language systems are coherent, natural kinds? Or do we imagine them to be? How could they cohere as systems through time, given that they are not replicated in confined events? Tree diagrams of language diversification are good for some things, but they are not good at representing the causal processes of language history, nor the true ontology of languages and language relatedness. The tree representation assumes that we are primarily interested in one form of transmission of heritable characteristics---i.e., \textsc{vertical transmission} of features from a parent to a daughter language, normally involving child language acquisition. \textsc{Horizontal transmission}---i.e., transmission of features between languages whose speakers are in contact, normally involving adult language acquisition---is recognized but is regarded as noise that needs to be factored out from the vertical historical signal of primary interest. 

Language diversification does show some causal biases toward vertical transmission, and our research needs to determine the nature of these biases. But biases toward vertical transmission are significantly weaker than comparable biases in vertebrate lineages in biology. Historical processes in language differ in two fundamental ways from the kind of vertical transmission we see in the evolution of species in vertebrates. First, beside the vertical transmission that accounts for what is shared among daughter languages and parent populations, there is a significant amount of horizontal transmission, by which traits of a daughter language are neither inherited from a parent language nor internally innovated, but are adopted from a contact language. Nothing like this kind of horizontal transmission---at least not on anything remotely like this scale---occurs in the evolution of vertebrates. 

When confronted with horizontal transmission, many scholars of language change have looked for ways to distinguish it from a vertical signal, and, usually, to exclude it. But the fact that horizontal transmission is so ubiquitous should instead have caused people to question the appropriateness of a model in which vertical transmission is the  main concern in representing and understanding language history. With a proper understanding of the causality of language change, we see that tree diagrams which take \textquoteleft the language' as the unit of analysis not only abstract from reality, they misrepresent it. They are poor conceptual tools for understanding the causal ontology of language. The solution is to get away from using \textquoteleft the language' as the evolutionarily basic causal unit in linguistic change, and to work with different units instead.

\section{The analogy with biology}
The kinds of life forms that have been implicitly taken to be the model for language---i.e., vertebrates---are quite unlike language in causal terms. Not only that, they are not particularly representative of life forms in general. Most forms of life, including not only the non-animal Eukaryotes, but also the Bacteria and Archaea, are not subject to strong vertical transmission constraints, since they do not have bounded body-plans that strictly delineate the structures and systems that can serve as vehicles or interactors for passing on replicable traits. Unlike vertebrates the overall phenotypic structures of \textquoteleft individuals' in many species are to a large degree emergent. 

The idea of languages as organisms has two sources: 1. idiolects as represented in brains (these are not reproduced in any single event; they change during lifetimes; they are never observed all at once; multiple such systems typically reside in one brain); 2. our meta awareness of community identity, along with sociometric watershed effects, cause us to think in essentialist terms about \textquoteleft languages', despite their obvious permeability.


What if we change our assumptions about the units involved? In Darwinian evolution, there must be a population of essentially equivalent units. These units must inherit characters from comparable units that existed prior to them. And these inheritable characters must show variation that can result in comparable units having different chances of surviving to pass on those characters to a new generation of units. What are these units? In the case of vertebrates, it has been widely assumed that two sorts of units work together: organisms, and genes. Organisms are vehicles for replication of genes. In the case of vertebrates, the vehicles for inheritance of characters are the bodies of individuals. Each body is a full phenotypic instantiation of the system. The situation with languages is not like this at all. 


When tree structures are used to represent the history of diversification within a family of languages, the standard interpretation of such trees is by direct analogy with the kind of evolution seen in life-forms that show a total or near-total bias toward vertical transmission in evolution, namely vertebrates such as primates, birds, fish, or reptiles. 

What does the tree diagram mean in the case of vertebrate natural history? Each binary branching in the tree represents a split in a breeding population, where the populations represented by the daughter nodes inherit a set of traits that were found in the parent population, but where members of the daughter populations also commonly inherit modifications of the parent traits that significantly distinguish the two daughter populations from each other. The causal process of inheritance is a single event of sexual reproduction in which a complete genotype is bestowed in the conception of a new individual. This encapsulation of the genome in inheritance is what ensures the vertical transmission that a tree diagram represents so well. In evolution of different vertebrate species, once two populations are no longer able to interbreed, they can no longer contribute to each other's historical gene pool. This would require \textit{horizontal transmission}, something that is essentially absent from vertebrate evolution (though with some minor caveats to be discussed below). The tree representation works adequately in this particular case for one reason: horizontal transmission is not captured. The vertebrate genome is essentially passed on as a bundle, and can thus be treated as a unit. The vehicle for replication is the individual organism as defined by the structurally coherent entity that we refer to as the body.

\section{Thinking causally about language change}
We want a causal account of languages as historically evolved systems. All the bits of language you learned as an infant were created by enormous chains of social interaction in the long history of a community. The cycle of transmission goes from public (someone uses a structure when speaking) to private (a second person's mental state is affected, when the structure is learnt or further entrenched) and back to public (the second person uses the structure, exposing someone else), and so on. Causal statements about language often highlight just part of what is going on. Consider (1) and (2):


\begin{quotation}
(1) \textit{Knowledge of grammar causes instances of speaking.} 
\end{quotation} 
\begin{quotation}
(2) \textit{Instances of speaking cause knowledge of grammar. }
\end{quotation} 

Statement (1) emphasizes competence. It points to mechanisms of, and prerequisites for, language production. Statement (2) emphasizes the outcomes of performance. 


This points to matters of language comprehension, learning, and cultural history. But there is no dichotomy here. The statements shown in (1) and (2) are ways of framing the same phenomenon. Competence and performance are equally indispensable in the population-level processes of historical evolution that determine and constrain what a language can ultimately look like. Words are effectively competing for our selection (REFS), and in this context an economy of words arises, with a host of consequences (Zipf 1949). What drives that selection is the efficacy of our choice in manipulating other people's attentional and interpretive resources. 


\section{Linguistic system}


What we usually refer to as languages or systems can also be thought of as focussed bundles of items. On this view, the \textquoteleft system' is not a natural kind. By the organism metaphor, \textquoteleft a language' is a thing, with some slow evolution, and with descendants. This kind of idea was challenged a long time ago in anthropology. Of Radcliffe-Brown and students (Fortes and Evans-Pritchard 1940), Leach wrote that \textquoteleft [s]ocial systems were spoken of as if they were naturally existing real entities and the equilibrium inherent in such systems was intrinsic' (Leach 1964:x), and that \textquoteleft I do not consider that social systems are a natural reality. In my view, the facts of ethnography and of history  can only \textit{appear} to be ordered in a systematic way if we impose upon these facts a figment of thought' (Leach 1964:xii). The prior unit is the \textsc{item}. Both linguists and speakers extrapolate the idea of systems (usually on the basis of a very few salient diagnostic items; cf. LePage and Tabouret-Keller 1985). We can take an emergent approach to system as depending on a concept of independent system, and the co-operation of social associates in acting as if the system were empirically real---this can be explained by social intelligence.

The macro-level facts about languages which are basic in linguistics (e.g. the things we see written in grammars, dictionaries, and historical reconstructions) are directly derived from the micro-level facts of individual embodiment of signs, and individual linguistic behaviour. A hypothesis is that the macro-level facts are \textit{derived} from the micro-level facts, by abstraction (at various levels/degrees), on the part of both normal speakers, and linguists. 


on linguistic item: \textquoteleft ...We need to distance ourselves somewhat from the concepts represented by the words \textit{language} and \textit{dialect}, which are a reasonable reflection of our lay culture, called \textquoteleft common-sense knowledge' (see 3.1.1), but not helpful in sociolinguistics. First, we need a term for the individual \textquoteleft bits of language' to which some sociolinguistic statements need to refer, where more global statements are not possible. We have already used the term \textsc{linguistic item} (2.1.1) and shall continue to use it as a technical term.' (Hudson 1980:22)







A prerequisite to the idea of \textquoteleft a language' (e.g. \textquoteleft English') is the idea of a specific group of people who speak it. As Le Page and Tabouret-Keller put it, \textquoteleft groups or communities and the linguistic attributes of such groups have no existential locus other than in the minds of individuals.' (1985:4.) 

 \textquoteleft We do not ourselves then need to put a boundary around any group of speakers and say \textquotedblleft These are the speakers of Language A, different from Language B", except to the extent that the people think of themselves in that way, and identify with or distance themselves from others by their behaviour' (Le Page and Tabouret-Keller 1985:9). 


Now note that people who identify differently are not necessarily in separate social systems, but may maintain ongoing social association between those different identity groups, with structured communicative exchange. The second is that social identities are prone to fluidity and change. In the hills of Northeast Burma, Leach (1964) studied social relations between speakers of Jinghpaw (Tibeto-Burman) and speakers of Shan (Tai), and the fluidity of their identities in terms of race, language, social label, and political position. 
 
\begin{quotation}
In any geographical area which lacks fundamental natural frontiers, the human beings in adjacent areas of the map are likely to have relations with one another - at least to some extent - no matter what their cultural attributes may be. In so far as these relations are ordered and not wholly haphazard there is implicit in them a social structure. But, it may be asked, if social structures are expressed in cultural symbols, how can the structural relations between groups of different culture [sic] be expressed at all? My answer to this is that the maintenance and insistence upon cultural difference can itself become a ritual action expressive of social relations. (Leach 1964:17)


\end{quotation}




It has always been recognised that an important place to look in trying to understand how the linguistic habits of identified groups change over generations is in the process of language acquisition, and the slight differences in habits of speech between children and their parents. Metaphorically speaking, language acquisition involves �transmission� of �a language� from parents to children. Imperfections in this transmission are held to account for language change. Strikingly consistent patterns in the details of such changes have been documented across a wide range of the world�s languages, leading many to argue that there are natural paths of semantic change which find their source in species-wide innate conceptual structure. While there are undoubtedly some universals in semantic change, which logically seek an explanation independent of social factors, and other factors outside the minds/brains/bodies of speakers, this is only part of the story. Even when new ideas for ways of saying things have their source within an individual, the spread of that idea to others follows the epidemiological pattern outlined here.
What is referred to by the terms �transmission� can be usefully understood as analogous to epidemiology (with acknowledgement to Sperber 1985). That is, we �catch� ideas from others, in this case ideas for attributing meanings to particular mediating structures.


\begin{quotation}
	An innovation in a language begins its existence in the mouths and minds of one or more speakers and spreads from them to other speakers. In fact, innovations occur constantly in the speech of individuals, but an innovation becomes part of the history of the language only when it spreads through the network to become a stable feature in the speech of a group of speakers. (Ross 1997:214-5)

\end{quotation}
	
	Narrowing in on syntax, Harris and Campbell make a similar observation:

\begin{quotation}
	We suggest that isolated creative, exploratory expressions are made constantly by speakers of all ages. Such expressions may be developed for emphasis, for stylistic or pragmatic reasons (to facilitate communication as in changes to avoid ambiguity or to foster easier identification of discourse roles), or they may result from production errors. The vast majority of such expressions are never repeated, but a few \textquoteleft catch on'. (Harris and Campbell 1995:54)
\end{quotation}

	Like a joke that someone makes up , and that spreads throughout the population - \textit{how} does it make this leap from the mind and mouth of a single speaker to the status of a stable feature? The process of an innovation catching on, or becoming a stable feature in the speech of a group of speakers is laid out in a chapter below� (XREF).


\begin{list}{label}{spacing}
\item (a) 	being exposed to semiotic material by which they may infer the idea; 

(b) 	being motivated to infer the idea, and to actually create a representation of the idea; 

(c) 	identifying as a person who would put the idea into practice;

(d) 	putting the idea into practice, exposing others (back to (a)). 

\end{list}





If it is the material restrictions of time, space, social group size and the need for convention that create the very high levels of convergence in the linguistic behaviour and complex sign systems of social associates, then what accounts for the �system properties� of language, such as Greenbergian implicational universals? \textit{(The same question may be asked in anthropology generally, but linguistic anthropology allows us greater control.)} 


Mead (1934:133) said �The processes of experience which the human brain makes possible are made possible only for a group of interacting individuals: only for individual organisms which are members of a society; not for the individual organism in isolation from other individual organisms�. (quoted in Rogoff 1990:14). See also Rosaldo (1984:145) for refs to Halliwell, Mead, Mauss and Fortes.

Seeing that it is now widespread, we assume it has spread widely. This implies a time depth for the spreading process, as well as a source and direction of the spread. There is evidence of a process which may be described as epidemiology (Sperber 1985). People catch ideas for ways of saying things. Factors familiar to us from the field of sociolinguistics have licensed and constrained the transmission and adoption of new ways of saying things from one group of people to another. The current synchronic state is the outcome of such a process.
	
The durability of signs is only maintained through constant field-testing of one's personal hypotheses about linguistic meaning. When you ask for \textit{salt} and get salt rather than pepper, your theory of the sign-category association <salt> is tested, confirmed, and therefore not revised. (There is no qualitative distinction, in this respect, between individual theories of the meanings of particular words, on the one hand, and grammatical categories, on the other.)



\textit{The study of areal phenomena both force us to consider, and help us to answer, the central question of linguistics: What is the nature of language? What is the ontology of language?}



