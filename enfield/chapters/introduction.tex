\chapter{Preface}


\begin{quotation}
An especially powerful form for theory is a body of underlying 
mechanisms, whose interactions and compositions provide the answers to 
all the questions we have. (Newell 1990:14)
\end{quotation}

A long-standing challenge for linguists is to discover ways to tell why neighboring languages can share structure. Grappling with this challenge has led me to consider conceptual questions at the foundations of language science, and these are the questions I want to raise in this book. They are questions of causality. What makes languages the way they are? In the analysis of language contact, maintenance, and change, I have found a certain set of ideas to be indispensable. Together, they form a \textsc{biased transmission} framework for understanding the causal foundations of language. This interdisciplinary framework draws not only on the linguistics of descriptive grammar, semantics, pragmatics, and typology, but also on sociological research on innovation diffusion, sociolinguistic research on social networks, and the natural science of cultural evolution. \footnote{On innovation diffusion, see Rogers (2003); on language and social networks, see Milroy (1980), Le Page and Tabouret-Keller (1985), Ross (1997), and Nettle (1999); on cultural evolution, see Sperber (2006), Boyd and Richerson (2005)}The framework is grounded in a commitment to being maximally explicit about the causal processes of social diffusion of linguistic and other cultural forms. The framework is built upon two main conceptual components, which are explicated later in the book.

1. \textsc{Interlocking causal frames}: There are multiple frames or time-scales at which  change in language and social relations can be causally affected. Most approaches work with one or two of these frames. The framework to be developed in this book insists on simultaneous attention to all of these frames, and special attention to the links between them. As explicated in Chapter 2, the framework recognizes six such frames: \textsc{microgenetic} (invoking cognitive and motoric processes involved in producing and comprehending language), \textsc{ontogenetic} (invoking lifespan processes by which people, usually as children, acquire linguistic knowledge and skills), \textsc{phylogenetic} (invoking ways in which the requisite cognitive capacities have evolved in our species), \textsc{enchronic} (invoking the sequential interlocking of social actions in linguistic clothing), \textsc{diachronic} (invoking historical change, conducted socially in human populations), and \textsc{synchronic} (any approach that does not explicitly invoke notions of process, such as language description).

2. \textsc{Transmission biases}: A socially- and cognitively-grounded theory of the diffusion of innovations in human populations provides the causal basis for how it is that social conventions (such as the ethnographic and linguistic facts that we observe) are the way they are. As explicated in Chapter 3, the causal machinery for diffusion of types of behavior (including language) within a society is a forward-feeding chain with four linked loci: \textsc{perception} of a bit of behaviour, \textsc{representation} of that bit of behaviour, subsequent \textsc{production} of that bit of behaviour, and \textsc{instantiation} of some trace of the behavior (leading to perception by others, feeding into the process anew). Each locus is a site where the chain of diffusion may be broken or reinforced: such breaks and reinforcements come from biases that may operate on each locus (Chapter 3 gives the details). There are many such biases. Some are cognitive. For example, if a linguistic construction is easier to learn, it will diffuse better. Some are social: If a more prestigious group of people make an innovation, it is more likely to be copied. While many of these biases are well known, the framework to be outlined below strives to build a testable conceptual account of the biases on social transmission.\footnote{Some approaches work with related ideas using computational methods such as agent-based modeling and bioinformatics, and applying the concepts of evolutionary biology (e.g., Kirby et al 2008, Dunn et al 2011). The present framework is intended to have potential for a direct interface with those lines of research, as the development of the causal account proposed here is at the conceptual level, and is independent from the specific methods used. While the application of bioinformatic methods is beyond the scope of this work, the framework and findings would ideally lead to close exchange with those approaches.}

