

\textbf{REFERENCES} (to be edited)





Adam, Barbara. 1990. \textit{Time and Social Theory}. Philadelphia: 
Temple Univ Pr.

Adams, Douglas, and Lloyd, John. 1983. \textit{The meaning of Liff}. 
London: Pan, Faber and Faber.

Agha, Asif. 2007. \textit{Language and Social Relations}. Cambridge: 
Cambridge University Press.

Aikhenvald, Alexandra Y., and R. M. W. Dixon. 1998. "\,Dependencies 
Between Grammatical Systems." \textit{Language} 74 (1): 56-80.

Atkinson, J. Maxwell, and John Heritage. 1984. \textit{Structures of 
Social Action: Studies in Conversation Analysis}. Cambridge: Cambridge 
University Press.

Aunger, Robert. 2000. \textit{Darwinizing Culture: The Status of 
Memetics as a Science}. Oxford: Oxford University Press.

Aureli, F., C. M. Schaffner, C. Boesch, S. K. Bearder, J. Call, C. A. 
Chapman, R. Connor, A. Di Fiore, R. I. M. Dunbar, and S. P. Henzi. 2008. 
"\,Fission-Fusion Dynamics." \textit{Current Anthropology} 49 (4): 
627-654.

Austin, J.L. 1962. \textit{How to do things with words}. Oxford 
University Press.

Baddeley, Alan D. 1986. \textit{Working Memory}. Oxford: Clarendon 
Press.

Bakhtin, 1981.

Bisang, 1991.

Bloch, Maurice. 1977. "\,The Past and the Present in the Present." 
\textit{Man (N. S.)} 12 (2): 278-292.

Bloch, Maurice. 2000. "\,A Well-disposed Social Anthropologist's 
Problems with Memes." In \textit{Darwinizing Culture: The Status of 
Memetics as a Science}, edited by Robert Aunger. OUP Oxford.

Bloomfield, Leonard. 1933. \textit{Language}. New York: Holt.

Bourdieu, Pierre. 1977. \textit{Outline of a Theory of Practice}. 
Cambridge: Cambridge University Press.

Boyd, Robert, and Peter J. Richerson. 1985. \textit{Culture and the 
Evolutionary Process}. Chicago: University of Chicago Press.

Boyd, Robert, and Peter J. Richerson. 2005. \textit{The Origin and 
Evolution of Cultures}. New York: Oxford University Press.

Buvac, Sasa. 1995. Resolving lexical ambiguity using a formal theory of 
context. In van Deemter and Peters (eds), 101--124.

Bybee, Joan. 2010. \textit{Language, Usage and Cognition}. 1 edition. 
Cambridge University Press.

Casson, R. W. Schemata in cognitive anthropology. \textit{Annual review 
of anthropology} 12, 429-462.

Chafe, Wallace. 1994. \textit{Discourse, Consciousness, and Time: The 
Flow and Displacement of Conscious Experience in Speaking and Writing}
. Chicago: University of Chicago Press.

Chamberlain, 2000.

Chater, Nick, and Morten H. Christiansen. 2010. "\,Language Acquisition 
Meets Language Evolution." \textit{Cognitive Science} 34 (7): 1131-57. 
doi:10.1111/j.1551-6709.2009.01049.x.

Chomsky, Noam A. 1965. \textit{Aspects of the theory of syntax}. 
Cambridge, Mass.: MIT Press.

Chomsky, Noam A. 2011. "\,Language and Other Cognitive Systems: What Is 
Special About Language?" \textit{Language Learning and Development} 7 
(4): 263-278. doi:10.1080/15475441.2011.584041.

Christiansen, Morten H., and Nick Chater. 2008. "\,Language as Shaped by 
the Brain." \textit{Behavioral and Brain Sciences} 31 (5): 489-509.

Clark, Eve V. 2009. \textit{First Language Acquisition}. Cambridge 
University Press.

Clark, Herbert H. 1996. Communities, commonalities, and communication. 
In Gumperz and Levinson (eds), 324--355.

Clark, Herbert H., and Barbara C. Malt. 1984. "\,Psychological 
Constraints on Language: a Commentary on Bresnan and Kaplan and on 
Giv�n." In \textit{Methods and Tactics in Cognitive Science}, edited 
by Walter Kintsch, James R. Miller, and Peter G. Polson, 191-214. 
Hillsdale, NJ: Lawrence Erlbaum.

Cole, M. 2007. "\,Phylogeny and Cultural History in Ontogeny." \textit{
Journal of Physiology-Paris} 101 (4): 236-246.

Cole, M. 1996. \textit{Cultural Psychology: A Once and Future 
Discipline}. Harvard University Press.

Croft, William. 2000. \textit{Explaining Language Change: An 
Evolutionary Approach}. Harlow: Longman.

Cutler, Anne. 2012. \textit{Native Listening: Language Experience and 
the Recognition of Spoken Words}. MIT Press.

D'Andrade, Roy D. 1987. A folk model of the mind. In Dorothy Holland and 
Naomi Quinn (eds). \textit{Cultural models in language and thought}. 
Cambridge University Press, 112--150.

D'Andrade, Roy D. 1992. Cognitive anthropology. In \textit{New 
directions in psychological anthropology}, edited by Theodore 
Schwartz, Geoffrey M. White, and Catherine A. Lutz. Cambridge: CUP, 
47-58.

Darwin, Charles. 1859. \textit{On the Origin of Species by Means of 
Natural Selection}. London: John Murray.

Darwin, Charles. 1871. \textit{The Descent of Man, and Selection in 
Relation to Sex}. London: John Murray.

Darwin, Charles. 1872. \textit{The Expression of the Emotions in Man 
and Animals}. London: J. Murray.

Davidson, Donald. 2006. \textit{The Essential Davidson}. Oxford: 
Clarendon.

Dawkins, Richard. 1976. \textit{The Selfish Gene}. Oxford: Oxford 
University Press.

Dawkins, Richard. 1982. \textit{The Extended Phenotype: The Long Reach 
of the Gene}. Oxford University Press.

Deemter, Kees van, and Peters, Stanley. 1995. \textit{Semantic 
ambiguity and underspecification}. Stanford: CSLI Publications.

Dennett, Daniel C. 1995. \textit{Darwin's dangerous idea: evolution and 
the meanings of life}. London: Penguin Books.

Dixon, R.M.W. 1997. \textit{The Rise and Fall of Languages}. 
Cambridge: Cambridge University Press.

Donald, Merlin. 2007. "\,The Slow Process: A Hypothetical Cognitive 
Adaptation for Distributed Cognitive Networks." \textit{Journal of 
Physiology, Paris} 101 (4-6): 214-222.

Donegan, and David Stampe. 2002. "\,South-East Asian Features in the 
Munda Languages: Evidence for the Analytic-to-synthetic Drift of Munda." 
In \textit{Proceedings of the 28th Annual Meeting of the Berkeley 
Linguistics Society, Special Session on Tibeto-Burman and Southeast 
Asian Linguistics, in Honor of Prof. James A. Matisoff}, edited by 
Patrick Chew, 111-129. Berkeley, CA: Berkeley Linguistics Society.

Donegan, Jane, and David Stampe. 1983. "\,Rhythm and the Holistic 
Organization of Language Structure." In \textit{The Interplay of 
Phonology, Morphology, and Syntax}, edited by John F. Richardson, 
Mitchell Marks, and Amy Chukerman, 337-353. Chicago: Chicago Linguistic 
Society.

Dunbar, Robin. 1992.

Dunbar, Robin. 1996. \textit{Grooming, gossip and the evolution of 
language.} London: Faber and Faber.

Duranti, Allesandro. 1997. \textit{Linguistic anthropology}. 
Cambridge University Press.

Durham, 1991.

Durham, William H. 1991. \textit{Coevolution}. Stanford University 
Press.

Durie, and Ross. 1996.

Durkheim, Emile. 1912. \textit{The Elementary Forms of the Religious 
Life}. Oxford University Press.

Eckert, Penelope. 2000. \textit{Linguistic Variation as Social Practice
}. Oxford: Blackwell.

Eckert, Penelope. 2008. "\,Variation and the Indexical Field." \textit{
Journal of Sociolinguistics} 12 (4): 453-476. 
doi:10.1111/j.1467-9841.2008.00374.x.

Emmorey, Karen. 2002. \textit{Language, Cognition and the Brain: 
Insights from Sign Language Research}. Mahwah, NJ: Lawrence Erlbaum.

Enfield, N. J. 2000a. The theory of cultural logic: how individuals 
combine social intelligence with semiotics to create and maintain 
cultural meaning. \textit{Cultural Dynamics}, 12.1, 35-64.

Enfield, N. J. 2002. \textit{Ethnosyntax: Explorations in Culture and 
Grammar}. Oxford: Oxford University Press.

Enfield, N. J. 2003. \textit{Linguistic Epidemiology: Semantics and 
Grammar of Language Contact in Mainland Southeast Asia}. London: 
RoutledgeCurzon.

Enfield, N. J. 2005. "\,Areal Linguistics and Mainland Southeast Asia." 
\textit{Annual Review of Anthropology} 34: 181-206.

Enfield, N. J. 2005. "\,The Body as a Cognitive Artifact in Kinship 
Representations. Hand Gesture Diagrams by Speakers of Lao." \textit{
Current Anthropology} 46 (1): 51-81.

Enfield, N. J. 2007. \textit{A Grammar of Lao}. Berlin: Mouton de 
Gruyter.

Enfield, N. J. 2008. "\,Transmission Biases in Linguistic Epidemiology." 
\textit{Journal of Language Contact} THEMA 2: 295-306.

Enfield, N. J. 2009. \textit{The Anatomy of Meaning: Speech, Gesture, 
and Composite Utterances}. Cambridge: Cambridge University Press.

Enfield, N. J. 2011. "\,Linguistic Diversity in Mainland Southeast 
Asia." In \textit{Dynamics of Human Diversity: The Case of Mainland 
Southeast Asia}, edited by N. J. Enfield, 63-80. Canberra: Pacific 
Linguistics.

Enfield, N. J. 2013. \textit{Relationship Thinking: Agency, Enchrony, 
and Human Sociality}. New York: Oxford University Press.

Enfield, N. J., and Jack Sidnell. 2014. "\,Language Presupposes an 
Enchronic Infrastructure for Social Interaction." In \textit{The Social 
Origins of Language}, edited by Daniel Dor, Chris Knight, and J. 
Lewis. Oxford: Oxford University Press.

Enfield, N. J., and Stephen C. Levinson, (eds) 2006.\textit{ Roots of 
Human Sociality: Culture, Cognition, and Interaction}. London: Berg.

Enfield, N. J., Mark Dingemanse, Julija Baranova, Joe Blythe, Penelope 
Brown, Tyko Dirksmeyer, Paul Drew, et al. 2013. "\,Huh? What? - A First 
Survey in 21 Languages." In \textit{Conversational Repair and Human 
Understanding}, edited by Makoto Hayashi, Geoffrey Raymond, and Jack 
Sidnell, 343-380. Cambridge: Cambridge University Press.

Enfield, N.J. to appear. Linguocentrism. In \textit{Explorations in 
linguistic relativity}, edited by Martin P�tz and M.H. Verspoor.

Evans-Pritchard, E. E. 1939. "\,Nuer Time-Reckoning." \textit{Africa} 
12 (02): 189-216. doi:10.2307/1155085.

Evans-Pritchard, E. E. 1940. \textit{The Nuer: A Description of the 
Modes of Livelihood and Political Institutions of a Nilotic People.} 
Clarendon Press.

Evans, Grant. 1993. Introduction. In Grant Evans (ed) 1993. \textit{
Asia's cultural mosaic: an anthropological introduction}. Singapore: 
Prentice Hall.

Evans, Nicholas D. 2012. "\,An Enigma Under an Enigma: Unsolved 
Linguistic Paradoxes in a Sometime Continent of Hunter-gatherers." Talk 
given in Amsterdam.

Firth, Raymond (1972). \textit{Verbal and body rituals of greeting and 
parting.} In \textit{The Interpretation of rituals}, edited by La 
Fontaine. London: Tavistock.

Firth, Raymond. 1936. \textit{We The Tikopia: A Sociological Study Of 
Kinship In Primitive Polynesia}. London: Routledge.

Fodor, Jerry A. 1987. \textit{Psychosemantics}. Cambridge, MA: MIT 
Press.

Foley, William A. 1997. \textit{Anthropological linguistics: an 
introduction}. London: Blackwell.

Foley, William A., and Robert D. Van Valin Jr. 1984. \textit{Functional 
Syntax and Universal Grammar}. Cambridge: Cambridge University Press.

Fortes, Meyer. 1945. \textit{The Dynamics of Clanship Among the 
Tallensi}. Oxford University Press.

Fortes, Meyer. 1949. \textit{Social Structure.} Oxford: Clarendon 
Press.

Fraser, Helen. 1992. \textit{The subject of speech perception: an 
analysis of the philosophical foundations of the information-processing 
model of cognition.} London: Macmillan.

Fraser, Helen. 1996. The subject in linguistics. In \textit{Language 
and the subject}, edited by K. Simms. Rodopi, 115-125.

Gabelentz, Georg von der. 1891. \textit{Die Sprachwissenschaft, Ihre 
Aufgaben, Methoden Und Bisherigen Ergebnisse}. 2nd ed. London: 
Routledge/Thoemmes Press.

Garfinkel, Harold. 1952. \textit{The Perception of the Other: A Study 
in Social Order}. Harvard University.

Garfinkel, Harold. 1967. \textit{Studies in Ethnomethodology}. New 
Jersey: Prentice-Hall.

Gell, Alfred. 1992. \textit{The Anthropology of Time?: Cultural 
Constructions of Temporal Maps and Images / Alfred Gell}. Explorations 
in Anthropology, Explorations in Anthropology. Oxford?; Providence: 
Berg.

Gergely, Gy�rgy, and Gergely Csibra. 2006. "\,Sylvia's Recipe: The Role 
of Imitation and Pedagogy in the Transmission of Cultural Knowledge." In 
\textit{Roots of Human Sociality: Culture, Cognition, and Interaction}
, edited by N. J. Enfield and Stephen C. Levinson, 229-255. London: 
Berg.

Gibson, James J. 1979. \textit{The Ecological Approach to Visual 
Perception}. Boston: Houghton Mifflin.

Gigerenzer, Gerd, Ralph Hertwig, and Thorsten Pachur, ed. 2011. \textit{
Heuristics: The Foundations of Adaptive Behavior}. New York: Oxford 
University Press.

Giv�n, Talmy. 1984. \textit{Syntax: a Functional-typological 
Introduction}. Amsterdam Philadelphia: John Benjamins.

Gladwell, Malcolm. 2000. \textit{The tipping point: how little things 
can make a big difference}. Boston: Little, Brown.

Goddard, Cliff, and Wierzbicka, Anna (eds). 1994. \textit{Semantic and 
lexical universals}. Amsterdam: Benjamins.

Goffman, Erving. 1981b. \textit{Forms of Talk}. Philadelphia, PA: 
University of Pennsylvania Press.

Goffman, Erving. 1967. \textit{Interaction ritual: essays on 
face-to-face behaviour}. Penguin.

Goffman, Erving. 1971. \textit{Relations in public: microstudies of the 
public order}. Penguin.

Goffman, Erving. 1981a. "\,Replies and Responses." In \textit{Forms of 
Talk}. University of Pennsylvania Press.

Good, David. 1995. Where does foresight end and hindsight begin? In 
Goody (ed), 139--149.

Goodwin, Charles. 2000. "\,Action and Embodiment Within Situated Human 
Interaction." \textit{Journal of Pragmatics} 32: 1489-1522.

Goodwin, Charles. 2002. "\,Time in Action." \textit{Current 
Anthropology} 43: S19-S35.

Goodwin, Charles. 2006. "\,Human Sociality as Mutual Orientation in a 
Rich Interactive Environment: Multimodal Utterances and Pointing in 
Aphasia." In \textit{Roots of Human Sociality: Culture, Cognition, and 
Interaction}, edited by N. J Enfield and Stephen C. Levinson, 97-125. 
London: Berg.

Goody, Esther. (ed) 1995. \textit{Social intelligence and interaction: 
expressions and implications of the social bias in human intelligence}
. Cambridge University Press.

Goody, Esther. 1995. Social intelligence and prayer as dialogue. In 
Goody (ed), 206--220.

Goody, Jack. 1991. "\,Time: Social Organization $[$1968$]$." In \textit{
International Encyclopedia of Social Sciences}, edited by D.L. Sills, 
16:30-42. New York: Macmillan.

Gould, Stephen Jay. 1977. \textit{Ontogeny and Phylogeny}. Harvard 
University Press.

Granovetter, Mark. 1973. "\,The Strength of Weak Ties." \textit{
American Journal of Sociology} 78: 1360-1380.

Green, Georgia M. 1995. Ambiguity resolution and discourse 
interpretation. In van Deemter and Peters (eds), 1--26. 

Greenberg, Joseph H. 1966. "\,Some Universals of Grammar with Particular 
Reference to the Order of Meaningful Elements." In \textit{Universals 
of Language (second Edition)}, edited by Joseph H. Greenberg, 73-113. 
Cambridge, MA: MIT Press.

Grice, H. Paul. 1975. Logic and conversation. In \textit{Speech acts}
, edited by Peter Cole and Jerry L. Morgan. New York: Academic Press, 
41--58.

Gross, Maurice. 1975. \textit{M�thodes en syntaxe: r�gime des 
constructions compl�tives}. Paris: Hermann.

Gumperz, John J. 1982. \textit{Discourse Strategies}. Cambridge 
University Press.

Gumperz, John J., and Levinson, Stephen C. (eds) 1996. \textit{
Rethinking linguistic relativity}. Cambridge University Press.

Gurvitch, Georges. 1964. \textit{The Spectrum of Social Time}. 
Dordrecht: D. Reidel.

Hale, Kenneth L. 1986. "\,Notes on World View and Semantic Categories: 
Some Warlpiri Examples." In \textit{Features and Projections}, edited 
by Pieter Muysken and Henk van Riemsdijk, 233-254. Dordrecht: Foris.

Hallowel, 1955

Hanks, William F. 2010. \textit{Converting Words: Maya in the Age of 
the Cross}. Vol. 6. Univ of California Press.

Harris, Alice C., and Lyle Campbell. 1995. \textit{Historical Syntax in 
Cross-linguistic Perspective}. Cambridge: Cambridge University Press.

Haspelmath, Martin. 2004. "\,How Hopeless Is Genealogical Linguistics, 
and How Advanced Is Areal Linguistics?" \textit{Studies in Language} 
28 (1): 209-223.

Hauser, Marc D., Noam Chomsky, and W. Tecumseh Fitch. 2002. "\,The 
Faculty of Language: What Is It, Who Has It, and How Did It Evolve." 
\textit{Science} 298: 1569-1579.

Hedstr�m, Peter, and Richard Swedberg. 1998. \textit{Social Mechanisms: 
An Analytical Approach to Social Theory}. Cambridge: Cambridge 
University Press.

Heritage, John. 1984. \textit{Garfinkel and Ethnomethodology}. 
Cambridge: Polity Press.

Hill, Jane H. and Mannheim, Bruce. 1992. Language and world view. 
\textit{Annual review of anthropology}. 21, 381--406.

Hill, R. A., and Robin I. M. Dunbar. 2003. "\,Social Network Size in 
Humans." \textit{Human Nature} 14: 53-72.

Hockett, Charles F. 1987. \textit{Refurbishing our foundations: 
elementary linguistics from an advanced point of view}. Amsterdam and 
Philadelphia: John Benjamins.

Holland, Dorothy, and Quinn, Naomi. (eds) 1987.\textit{Cultural models 
in language and thought}. Cambridge University Press.

Hopper, Paul J., and Elizabeth Closs Traugott. 1993. \textit{
Grammaticalization}. Cambridge: Cambridge University Press.

Hopper, Paul. J. 1987. Emergent grammar. \textit{Berkeley Linguistics 
Society, proceedings of the 13th meeting}, 139--157.

Horton, R. 1982. Tradition and modernity revisited. In \textit{
Rationality and relativism}, edited by M. Hollis and S. Lukes. Oxford 
University Press.

Hudson, R. A. 1996. \textit{Sociolinguistics (Second Edition)}. 
Cambridge: Cambridge University Press.

Hughes, Diane Owen, and Thomas R. Trautmann, ed. 1995. \textit{Time: 
Histories and Ethnologies}. University of Michigan Press.

Hurford, James R. 2007. \textit{The Origins of Meaning}. Oxford: 
Oxford University Press.

Hurford, James R. 2012. \textit{The Origins of Grammar}. Oxford: 
Oxford University Press.

Hutchins, Edwin, and Hazlehurst, Brian. 1995. How to invent a shared 
lexicon: the emergence of shared form-meaning mappings in interaction. 
In Goody (ed), 53-67.

Hutchins, Edwin. 1980. \textit{Culture and inference: a Trobriand case 
study.} Cambridge, MA: Harvard University Press.

Hutchins, Edwin. 1995. \textit{Cognition in the wild}. Cambridge, 
Mass.: MIT Press.

Hutchins, Edwin. 2006. "\,The Distributed Cognition Perspective on Human 
Interaction." In \textit{Roots of Human Sociality: Culture, Cognition 
and Interaction}, edited by N. J. Enfield and Stephen C. Levinson, 
375-398. Oxford: Berg.

Keesing, Roger M. 1979. Linguistic knowledge and cultural knowledge: 
some doubts and speculations. \textit{American anthropologist}, 
81:1979, 14--36.

Keller, Rudi. 1994. \textit{On Language Change: The Invisible Hand in 
Language}. London/New York: Routledge.

Kirby, Simon, Hannah Cornish, and Kenny Smith. 2008. "\,Cumulative 
Cultural Evolution in the Laboratory: An Experimental Approach to the 
Origins of Structure in Human Language." \textit{Proceedings of the 
National Academy of Sciences of America} 105 (31): 10681-10686.

Kirby, Simon, Kenny Smith, and Henry Brighton. 2004. "\,From UG to 
Universals: Linguistic Adaptation Through Iterated Learning." \textit{
Studies in Language} 28 (3): 587-607. doi:10.1075/sl.28.3.09kir.

Kirby, Simon. 1999. \textit{Function, Selection, and Innateness: The 
Emergence of Language Universals}. Oxford: Oxford University Press.

Kirby, Simon. 2013. "\,Transitions: The Evolution of Linguistic 
Replicators." In \textit{The Language Phenomenon}, 121-138. Berlin, 
Heidelberg: Springer Verlag.

Kockelman, Paul. 2005. "\,The Semiotic Stance." \textit{Semiotica} 157 
(1/4): 233-304.

Kockelman, Paul. 2006. "\,Residence in the World: Affordances, 
Instruments, Actions, Roles, and Identities." \textit{Semiotica} 162 
(1-4): 19-71.

Kockelman, Paul. 2013. \textit{Agent, Person, Subject, Self: A Theory 
of Ontology, Interaction, and Infrastructure}. Oxford University 
Press.

Labov, William. 1986. "\,On the Mechanism of Linguistic Change." In 
\textit{Directions in Sociolinguistics: The Ethnography of 
Communication}, edited by John J. Gumperz and Dell Hymes, second 
edition:512-538. London: Basil Blackwell.

Labov, William. 2011. \textit{Principles of Linguistic Change, 
Cognitive and Cultural Factors}. John Wiley \& Sons.

Langacker, Ronald W. 1987. \textit{Foundations of Cognitive Grammar: 
Volume I, Theoretical Prerequisites}. Stanford: Stanford University 
Press.

Langacker, Ronald W. 1990. \textit{Concept, image, and symbol: the 
cognitive basis of grammar}. Berlin and New York: Mouton de Gruyter.

Langacker, Ronald W. 1994. Culture, cognition, and grammar. In \textit{
Language contact language conflict}, edited by Martin P�tz, 
Amsterdam/Philadelphia: John Benjamins, 25--53.

Larsen-Freeman, D., and D. Cameron. 2008. \textit{Complexity Theory and 
Second Language Learning}. Oxford University Press.

Laver, John. 1974. Communicative functions of phatic communion. In 
\textit{Organization of behaviour in face-to-face interaction}, 
edited by Kendon, Harris and Key. The Hague: Mouton.

Laver, John. 1981. Linguistic Routines and Politeness in Greeting and 
Parting. In \textit{Conversational Routine; Explorations in 
standardized communication situations and prepatterned speech}, edited 
by Florian Coulmas. The Hague: Mouton.

Layton, Robert. 1997. \textit{An introduction to theory in anthropology
}. Cambridge University Press.

Le Page 1968.

Le Page, R. B., and Andr�e Tabouret-Keller. 1985. \textit{Acts of 
Identity: Creole-based Approaches to Language and Ethnicity}. 
Cambridge: Cambridge University Press.

Leach, 1964.

Leach, Edmund. 1961. \textit{Rethinking Anthropology}. London: The 
Athlone Press.

Leach, Edmund. 1976. \textit{Culture and communication}. Cambridge 
University Press.

LeDoux, Joseph. 1998. \textit{The emotional brain}. London: Phoenix.

Lee, Penny. 1996. \textit{The Whorf theory complex: a critical 
reconstruction}. Amsterdam: John Benjamins.

Lemke, J. L. 2002. "\,Language Development and Identity: Multiple 
Timescales in the Social Ecology of Learning." \textit{Language 
Acquisition and Language Socialization: Ecological Perspectives}: 
68-87.

Lemke, J. L. 2000. "\,Across the Scales of Time: Artifacts, Activities, 
and Meanings in Ecosocial Systems." \textit{Mind, Culture, and Activity
} 7 (4): 273-290.

Leont'ev, A. 1981. \textit{Problems of the development of mind}. 
Moscow (Russian original 1947): Progress Press.

Levelt, Willem J. M. 1989. \textit{Speaking: From Intention to 
Articulation}. Cambridge, MA: MIT Press.

Levelt, Willem J. M. 2012. \textit{A History of Psycholinguistics: The 
Pre-Chomskyan Era}. Oxford University Press.

Levinson, Stephen C. 1983. \textit{Pragmatics}. Cambridge: Cambridge 
University Press.

Levinson, Stephen C. 1995. Interactional biases in human thinking. In 
Goody (ed), 221--260.

Lewis, D. K. 1969. \textit{Convention: a philosophical study}. 
Harvard University Press.

Locke, . 1961$[$1690$]$

Lorenz, Konrad Z. 1958. "\,The Evolution of Behavior." \textit{
Scientific American} 199 (6): 67-83.

Luce, R. D. 1950. Connectivity and generalized cliques in sociometric 
group structure. \textit{Psychometrika}, 15, 169-190.

Lucy, John. 1992. \textit{Language Diversity and Thought: a 
Reformulation of the Linguistic Relativity Hypothesis}. Cambridge: 
Cambridge University Press.

Lumsden and Wilson 1981.

MacNeilage, Peter F. 1998. "\,The Frame/content Theory of Evolution of 
Speech Production." \textit{Behavioral and Brain Sciences} 21 (04): 
499-511.

Macwhinney, B. 2005. "\,The Emergence of Linguistic Form in Time." 
\textit{Connection Science} 17 (3-4): 191-211.

Malinowski, Bronislaw. 1922. \textit{Argonauts of the Western Pacific: 
An Account of Native Enterprise and Adventure in the Archipelagoes of 
Melanesian New Guinea}. London: Routledge.

Manning, Patrick. 2005. \textit{Migration in World History}. New York 
and London: Routledge.

Marx, Karl, and Friedrich Engels. 1947. \textit{The German Ideology}. 
New York: International Publishers.

Mayr, Ernst. 1982. \textit{The Growth of Biological Thought: Diversity, 
Evolution, and Inheritance}. Belknap Press.

McConvell, Patrick. 1985. "\,The Origin of Subsections in Northern 
Australia." \textit{Oceania} 56 (1): 1-33.

McNeill, David. 2005. \textit{Gesture and Thought}. Chicago and 
London: Chicago University Press.

Meillet, Antoine. 1926. \textit{Linguistique Historique Et Linguistique 
G�n�rale}. Paris: Champion.

Miller, George A. 1951. \textit{Language and communication}. New 
York: McGraw-Hill.

Milroy, Lesley, and Muysken, Pieter. (eds) 1995. \textit{One speaker, 
two languages: cross-disciplinary perspectives on code-switching}. 
Cambridge University Press.

Milroy, Leslie. 1980. \textit{Language and Social Networks}. Oxford: 
Basil Blackwell.

Mufwene, Salikoko S. 2001. \textit{The Ecology of Language Evolution}
. Cambridge: Cambridge University Press.

M�ller, Max. 1870. "\,Darwinism Tested by the Science of Language." 
\textit{Nature} 1 (10): 256-259.

Munn, Nancy D. 1992. "\,The Cultural Anthropology of Time: A Critical 
Essay." \textit{Annual Review of Anthropology} 21 (1): 93-123. 
doi:10.1146/annurev.an.21.100192.000521.

Nettle, Daniel. 1999. \textit{Linguistic diversity}. Oxford: Oxford 
University Press.

Newell, Allen. 1990. \textit{Unified Theories of Cognition}. Harvard 
University Press.

Norman, Donald A. 1991. "\,Cognitive Artifacts." In \textit{Designing 
Interaction: Psychology at the Human-computer Interface}, ed. John M. 
Carroll, 17-38. Cambridge: Cambridge University Press.

Pagel, Mark, Quentin D. Atkinson, and Andrew Meade. 2007. "\,Frequency 
of Word-use Predicts Rates of Lexical Evolution Throughout Indo-European 
History." \textit{Nature} 449: 717-720. doi:10.1038/nature06176.

Pagel, Mark, Quentin D. Atkinson, and Andrew Meade. 2007. "\,Frequency 
of Word-use Predicts Rates of Lexical Evolution Throughout Indo-European 
History." \textit{Nature} 449: 717-720. doi:10.1038/nature06176.

Parry, J., and Maurice Bloch, ed. 1989. \textit{Money and the Morality 
of Exchange}. Cambridge University Press.

Phillips, 1974.

Phillips, 1979.

Piantadosi, S.T., Harry Tily, and Edward Gibson. 2011. "\,Word Lengths 
Are Optimized for Efficient Communication." \textit{Proceedings of the 
National Academy of Sciences} 108 (9): 3526.

R�czaszek-Leonardi, J. 2010. "\,Multiple Time-Scales of Language 
Dynamics: An Example From Psycholinguistics." \textit{Ecological 
Psychology} 22 (4): 269-285.

Radcliffe-Brown, A. R. 1922. \textit{The Andaman Islanders a Study in 
Social Anthropology}. Cambridge University Press.

Radcliffe-Brown, A. R. 1931. "\,The Social Organization of Australian 
Tribes: Part III." \textit{Oceania} 1 (4): 426-456.

Reddy, Michael. 1979. The conduit metaphor --- a case of frame conflict 
in our language about language. In \textit{Metaphor and thought}, 
edited by\textit{ }Andrew Ortony. Cambridge University Press, 
284--324.

Ridley, Mark. 2004. \textit{Evolution}. Wiley.

Ridley, Mark. 1997. \textit{Evolution}. Oxford University Press.

Rigsby, B. and Sutton, P. 1982. Speech communities in aboriginal 
Australia. \textit{Anthropological Forum}, V.1, 8--23.

Rogers, Everett M. 1995. \textit{Diffusion of Innovations}. Vol. 
(Fourth edition.). New York: The Free Press.

Rogers, Everett M. 2003. \textit{Diffusion of Innovations}. (Fifth 
edition.). New York: The Free Press.

Rogoff, 1990.

Rosaldo, Michelle Z. 1984. Toward an anthropology of self and feeling. 
In \textit{Culture theory: essays on mind, self, and emotion}, edited 
by Richard A. Shweder and Robert A. LeVine, Cambridge: Cambridge 
University Press, 137-157.

Ross, Malcolm. 1997. Socials networks and kinds of speech-community 
event. In \textit{Archaeology and language I: theoretical and 
methodological orientations}. Edited by Roger Blench and Matthew 
Spriggs. London and New York: Routledge.

Runciman, W. G. 2009. \textit{The Theory of Cultural and Social 
Selection}. Cambridge: Cambridge University Press.

Sacks, Harvey, Emanuel A. Schegloff, and Gail Jefferson. 1974. "\,A 
Simplest Systematics for the Organization of Turn-taking for 
Conversation." \textit{Language} 50 (4): 696-735.

Sahlins, Marshall. 1999. "\,What Is Anthropological Enlightenment? Some 
Lessons of the Twentieth Century." \textit{Annual Review of 
Anthropology} 28 (1): i-xxiii. doi:10.1146/annurev.anthro.28.1.0.

Sapir, Edward, 1994. \textit{The psychology of culture}. 
(Reconstructed and edited by Judith T. Irvine.) Berlin, New York: Mouton 
de Gruyter.

Sapir, Edward. 1921. \textit{Language: An Introduction to the Study of 
Speech}. Orlando/San Diego/New York/London: Harcourt Brace Jovanovich.

Saussure, Ferdinand de. 1916. \textit{Cours De Linguistique G�n�rale}
. Paris: Payot.

Saussure, Ferdinand de. 1959. \textit{Course in General Linguistics}. 
New York: McGraw-Hill.

Schegloff, Emanuel A. 1968. "\,Sequencing in Conversational Openings." 
\textit{American Anthropologist} 70 (6): 1075-1095.

Schegloff, Emanuel A., Elinor Ochs, and Sandra A. Thompson. 1996. 
"\,Introduction." In \textit{Interaction and Grammar}. Cambridge 
University Press.

Schegloff, Emanuel A., Gail Jefferson, and Harvey Sacks. 1977. "\,The 
Preference for Self-correction in the Organization of Repair in 
Conversation." \textit{Language} 53 (2): 361-382.

Schelling, Thomas C. 1960. \textit{The strategy of conflict}. 
Cambridge, MA: Harvard University Press.

Schelling, Thomas C. 1978. \textit{Micromotives and Macrobehaviour}. 
New York: W. W. Norton.

Schieffelin, Bambi B., and Elinor Ochs. 1986. \textit{Language 
Socialization Across Cultures}. Cambridge: Cambridge University Press.

Schiffer, Stephen R. 1972. \textit{Meaning}. Oxford: Clarendon.

Schutz, Alfred. 1970. \textit{On Phenomenology and Social Relations}. 
Chicago: University of Chicago Press.

Searle, John R. 1983. \textit{Intentionality: An Essay in the 
Philosophy of Mind}. Cambridge: Cambridge University Press.

Searle, John R. 2010. \textit{Making the Social World: The Structure of 
Human Civilization}. New York: Oxford University Press.

Sidnell, Jack, and N. J. Enfield. 2012. "\,Language Diversity and Social 
Action." \textit{Current Anthropology} 53 (3): 302-333.

Sidnell, Jack, and Tanya Stivers, ed. 2012. \textit{The Handbook of 
Conversation Analysis}. Oxford: Wiley-Blackwell.

Simon, Herbert A. 1990. "\,A Mechanism for Social Selection and 
Successful Altruism." \textit{Science} 250: 1665-1668.

Slobin, Dan. 1996. "\,From 'Thought and Language' to 'Thinking to 
Speaking'." In \textit{Rethinking Linguistic Relativity}, edited by 
JJ Gumperz and Stephen C. Levinson, 70-96. Cambridge University Press.

Smith, Adam. 1776. \textit{An Inquiry into the Nature and Causes of the 
Wealth of Nations}. London: W. Strahan.

Smith, Kenny, Henry Brighton, and Simon Kirby. 2003. "\,Complex Systems 
in Language Evolution: The Cultural Emergence of Compositional 
Structure." In \textit{Advances in Complex Systems}, 537-558.

Sperber, Dan, and Dierdre Wilson. 1995. \textit{Relevance: 
Communication and Cognition}. 2nd edition. Oxford: Blackwell.

Sperber, Dan, and Lawrence A. Hirschfeld. 2004. "\,The Cognitive 
Foundations of Cultural Stability and Diversity." \textit{Trends in 
Cognitive Sciences} 8 (1): 40-46.

Sperber, Dan. 1985. "\,Anthropology and Psychology: Towards an 
Epidemiology of Representations." \textit{Man} 20: 73-89.

Sperber, Dan. 1996. \textit{Explaining Culture: a Naturalistic Approach
}. London: Blackwell.

Sperber. 2006. "\,Why a Deep Understanding of Cultural Evolution Is 
Incompatible with Shallow Psychology." In \textit{Roots of Human 
Sociality: Culture, Cognition, and Interaction}, ed. N. J. Enfield and 
Stephen C. Levinson, 431-449. Oxford: Berg.

Steels, Luc. 1998. "\,Synthesizing the Origins of Language and Meaning 
Using Coevolution, Self-organization and Level Formation." In \textit{
Approaches to the Evolution of Language: Social and Cognitive Bases}, 
edited by James R. Hurford, Michael Studdert-Kennedy, and Chris Knight. 
Cambridge: Cambridge University Press.

Steels, Luc. 2003. "\,Evolving Grounded Communication for Robots." 
\textit{Trends in Cognitive Sciences} 7: 308-312.

Steffenson, Sune V., and Alwin Fill. 2013. "\,Ecolinguistics: The State 
of the Art and Future Horizons." \textit{Language Sciences}.

Stivers, Tanya, Lorenza Mondada, and Jakob Steensig, ed. 2011. \textit{
The Morality of Knowledge in Conversation}. Cambridge: Cambridge 
University Press.

Strauss, Claudia, and Quinn, Naomi. 1997. \textit{A cognitive theory of 
cultural meaning}. Cambridge University Press.

Strauss, Claudia, and Quinn, Naomi. 1997. \textit{A cognitive theory of 
cultural meaning}. Cambridge: CUP.

Taylor, Charles. 1995. \textit{Philosophical arguments}. Harvard 
University Press.

Thomason, Sarah Grey, and Terrence Kaufman. 1988. \textit{Language 
Contact, Creolization, and Genetic Linguistics}. Berkeley: University 
of California Press.

Tomasello, Michael. 2003. \textit{Constructing a Language: a 
Usage-based Theory of Language Acquisition}. Cambridge, MA: Harvard 
University Press.

Tomasello, Michael. 2008. \textit{Origins of Human Communication}. 
Cambridge, MA: MIT Press.

Tomasello, Michael. 1999. \textit{The Cultural Origins of Human 
Cognition}. Cambridge, Mass: Harvard University Press.

Trudgill, Peter. 2010. "\,Contact and Sociolinguistic Typology." In 
\textit{The Handbook of Language Contact}, edited by Raymond Hickey. 
John Wiley \& Sons.

Uryu, Michiko, Sune V. Steffensen, and Claire Kramsch. 2013. "\,The 
Ecology of Intercultural Interaction: Timescales, Temporal Ranges and 
Identity Dynamics." \textit{Language Sciences $[$Special Issue on 
Ecolinguistics: The Ecology of Language and the Ecology of Science, Ed. 
by Sune Vork Steffensen and Alwin Fill$]$}

Varela, Francisco J, Thompson, Evan, and Rosch, Eleanor. 1991. \textit{
The embodied mind: cognitive science and human experience}. MIT Press.

Vygotsky, L. S. 1962. \textit{Thought and Language}. Cambridge, MA: 
MIT Press.

Ward, Barbara W. 1965. Varieties of the conscious model: the fishermen 
of South China. In \textit{The relevance of models for social 
anthropology}, edited by Michael Banton, London: Tavistock, 113-137.

Ward, Barbara. 1985. \textit{Through other eyes: an anthropologist's 
view of Hong Kong}. Hong Kong: Chinese University Press.

Weinreich, Uriel, William Labov, and Marvin Herzog. 1968. "\,Empirical 
Foundations for a Theory of Language Change." In \textit{Proceedings of 
the Texas Conference on Historical Linguistics}, edited by W. Lehmann, 
97-195. Austin: University of Texas Press.

Weinreich, Uriel. 1953. \textit{Languages in Contact}. New York: 
Linguistic Circle of New York.

Whorf, Benjamin Lee. 1956. \textit{Language, Thought, and Reality}. 
Cambridge, Mass: MIT Press.

Whorf.

Wierzbicka, Anna. 1988. \textit{The Semantics of Grammar}. Amsterdam: 
Benjamins.

Wierzbicka, Anna. 1992. \textit{Semantics, Culture, and Cognition}. 
New York: Oxford University Press.

Wierzbicka, Anna. 1996. \textit{Semantics: primes and universals}. 
New York: Oxford University Press.

Wierzbicka.

Wittgenstein, Ludwig. 1953. \textit{Philosophical investigations}. 
Oxford.

Zeitlyn, David. 1995. Divination as dialogue: negotiation of meaning 
with random responses. In Goody (ed), 189--205.

Zipf, G. K. 1935. \textit{The Psycho-biology of Language}. Boston: 
Houghton Mifflin.

Zipf, G. K. 1949. \textit{Human Behaviour and the Principle of Least 
Effort}. Cambridge, Mass: Addison-Wesley.





\newpage


\textbf{Acknowledgements}



In writing these chapters I have benefited greatly from conversations 
and correspondence with Stephen Cowley, Paul Kockelman, Michael Lempert, 
Joanna R�czaszek-Leonardi, Giovanni Rossi, Jack Sidnell, Chris Sinha, 
Kenny Smith, Sune Vork Steffensen, Jordan Zlatev, Chip Zuckerman, Morten 
Christiansen, Dan Dediu, Mark Dingemanse, Daniel Dor, Bill Hanks, 
Jennifer Johnson-Hanks, Simon Kirby, Chris Knight, Steve Levinson, Hugo 
Mercier, Pieter Muysken, Dan Sperber, Monica Tamariz, Csaba Pl�h, Rob 
Boyd, and Peter Richerson. I thank participants at the conference 
'Naturalistic Approaches to Culture' (Balatonvilagos, 2011), the 
conference 'Language, Culture, and Mind V' (Lisbon, 2012), the workshop 
'Rethinking Meaning' (Bologna, 2012), the 'Minerva-Gentner Symposium on 
Emergent Languages and Cultural Evolution' (Nijmegen, 2013) , and the 
conference 'Social Origins of Language' (London, 2011), for comments and 
reactions. This work is supported by the European Research Council 
(grant 'Human Sociality and Systems of Language Use', 2010-2014), and 
the Max Planck Institute for Psycholinguistics, Nijmegen. The chapters 
are revised versions of previously published works: as chapters in 
"\,Social Origins of Language" (OUP, edited by D. Dor, C. Knight, and J. 
Lewis), and \textit{Cambridge Handbook of Linguistic Anthropology} 
(CUP, 2014). This essay also draws on sections of Enfield (2008; 2013, 
chap. 11; 2014), and section 12.3.2 (pp358-365) of Enfield et al (2013). 






\end{document}