
\chapter{The Micro/Macro Solution}


While as members of a group we may feel subjectively convinced that 
there is a cultural totality around us, we never directly observe it. As 
Fortes put it, \textquoteleft structure is not immediately visible in the "\,concrete 
reality". It is discovered by comparison, induction and analysis based 
on a sample of actual social happenings in which the institution, 
organization, usage etc. with which we are concerned appears in a 
variety of contexts' (Fortes 1949, 56). This mode of discovery belongs 
not only to the ethnographer, but also to the child whose task is to 
become a competent adult (see Brown and Gaskins, 2014). 



If our experience of culture is in the micro, how, then, do we 
extrapolate to the macro? Parry and Bloch, in their discussion of money 
and its status, stress local differences between cultures and the 
effects on the meaning that money comes to have, but they acknowledge a 
certain unity across cultures. This unity is \textquoteleft neither in the meanings 
attributed to money nor in the moral evaluation of particular types of 
exchange, but rather in the way the totality of transactions form a 
general pattern which is part of the reproduction of social and 
ideological systems concerned with a time-scale far longer than the 
individual human life' (Parry and Bloch 1989, 1). 



In terms that apply more generally to the micro/macro issue, there is, 
they argue, \textquoteleft something very general about the relationship between the 
transient individual and the enduring social order which transcends the 
individual' (Parry and Bloch 1989, 2). It brings to mind Adam Smith's 
(1776, bk. 4, chap. 2) discussion of the relation between the 
motivations of individuals and the not-necessarily-intended 
population-level aggregate effects of their behavior (see also Schelling 
1978; Rogers 2003; Hedstr\"{o}m and Swedberg 1998). Parry and Bloch (1989, 
29) contrast \textquoteleft short-term order' with \textquoteleft long-term reproduction', and 
suggest that the two must be linked. 



This brings us back to the transmission criterion, a key issue for 
bridging the micro/macro divide. If the individual is to function as a 
member of a social group, he or she needs to construct, as an 
individual, the capacity to produce and properly interpret locally 
normative meaningful behaviour. 



Not even a cultural totality is exempt from the transmission criterion. 
Individual people must learn the component parts of a totality during 
their lifetimes, and they must be motivated to reproduce the behaviours 
that stabilize the totality and cause it to endure beyond their own 
lives and lifetimes. This motivation can of course be in the form of an 
salient external pressure such as the threat of state violence, but most 
often it is in the form of a less visible force, namely the regimenting 
currents of normative accountability (Enfield 2013, chapters 3, 9, and 
passim). 



We find ourselves in social/cultural contexts in which our behaviour 
will be interpreted as meaningful. \textquoteleft The big question is not whether 
actors understand each other or not', Garfinkel wrote (1952, 367; cited 
in Heritage 1984, 119). \textquoteleft The fact is that they do understand each other, 
that they \textit{will} understand each other, but the catch is that 
they will understand each other regardless of how they \textit{would} 
be understood.' So, if you are a member of a social group, you are not 
exempt from having others take your actions to have meanings, whether or 
not these were the meanings you wanted your actions to have. 



As Levinson (1983, 321) phrases it, also echoing Goffman and Sacks, we 
are \textquoteleft not so much constrained by rules or sanctions, as caught up in a 
web of inferences'. We will be held to account for others' 
interpretations of our behavior and we know this whether we like it or 
not. This is a powerful force in getting us to conform. Accountability 
to norms \textquoteleft constitutes the foundation of socially organized conduct as a 
self-producing environment of "\,perceivedly normal" activities' 
(Heritage 1984, 119). The thing that tells us what is \textquoteleft normal' is of 
course the culture. 



With respect to the production of normatively appropriate conduct, all 
that is required is that the actors have, and attribute to one another, 
a reflexive awareness of the normative accountability of their actions. 
For actors who, under these conditions, calculate the consequences of 
their actions in reflexively transforming the circumstances and 
relationships in which they find themselves, will routinely find that 
their interests are well served by normatively appropriate conduct. With 
respect to the anarchy of interests, the choice is not between 
normatively organized co-operative conduct and the disorganized pursuit 
of interests. Rather, normative accountability is the \textquoteleft grid' by 
reference to which \textit{whatever} is done will become visible and 
assessable. (Heritage 1984, 117)



One might ask what is \textquoteleft normatively appropriate conduct'. The answer must 
include any of the kinds of behaviours discussed in the above section on 
cultural systems: for example, behaving in accordance with the rules of 
a section system by marrying someone of the right category (or being 
able to produce an account for why one has done otherwise). They would 
not be cultural behaviours if they were not regimented in a community by 
accountability to norms (and laws). 



So the path that is both the least resistant and the most empowering for 
an individual is to learn the system that generates a shared set of 
normative interpretations of people's behaviour, and then go with the 
flow. This is how the totality cannot exist without the individuals, 
while-paradoxically-appearing to do just that. 



The close relationship between short-term order and long-term 
reproduction is an asymmetrical one. Short-term order is where the 
causal locus of transmission is found; it is where acceleration, 
deceleration, and transformation in cultural transmission occurs 
(Enfield 2014; cf. Schelling 1978; Rogers 2003; Sperber 1985; Sperber 
1996). 



From all of this it is clear that cultural systems exist and they both 
constrain us and influence us. The question is: How are systems \textit{
transmitted}? The regulation of individual behaviour in the cultural 
totality is not achieved by mere emergence like the behaviour of a bird 
in the seemingly concerted movement of a flock. Individuals' behaviour 
is regulated by norms, in an effectively telic way, and a good deal of 
cultural regimentation is done through explicit instruction, often with 
reference to norms, and sometimes to punishable laws. 



Where can we find an account of how whole systems are transmitted? As we 
saw in the last chapter, the only good causal account we have for social 
transmission through populations and across generations is one that 
works in terms of cultural items, not whole systems.



\section{The combinatoric nature of cultural items in general}


Recall that the context bias is grounded in the fact that one cannot 
behold any so-called item without beholding it \textit{in relation to }
something else, including not only things of similar kinds, but perhaps, 
especially, the social norms and intentions associated with items and 
the contexts in which they appear. So, I cannot know what a hammer is 
without seeing it in relation to, among other things, the human body, 
timber and nails, people's intentions to build things, conventional 
techniques for construction, and so on. 



These relations-which themselves are interrelated-form an indispensible 
part of what I'm referring to by the term \textquoteleft item'. When a cultural item 
diffuses, what is actually diffusing is something less like an object 
and more like a combinatoric relation. Thus, a hammer incorporates a 
handle or grip. The handle or grip has a combinatoric relation to the 
human hand insofar as the handle and the hand are practically and 
normatively designed to go together. The handle is designed that way 
because of how the human hand is. The handle only makes sense in terms 
of the hand. 



This going together of the handle and the hand is like a grammatical 
relation. In a similar way, the handle of the hammer and the head of the 
hammer are combinatorically related, again fitting together both 
practically and normatively. The head of the hammer stands, in turn, in 
a combinatoric relation to a nail. The nail, in turn, has a combinatoric 
relation to the timber, and so forth. So we see how the cultural \textquoteleft items' 
that diffuse in historical populations in interaction necessarily 
incorporate-and advertise-their combinatoric relations. 



The sprawling yet structured systems that we call languages have exactly 
the kinds of properties of incorporation and contextualisation that I 
have just described for concrete objects. Thus, if speakers of a 
language have borrowed a word from another language, they have not 
merely adopted a pairing of \textquoteleft sound image plus thought image', but must 
also have adopted a way of relating the word to the existing system 
(whether or not this relation resembles the one used in the source 
system). 



The word will not be usable if it does not have combinatoric properties. 
The norms for combining the word in usage may be borrowed along with the 
word itself, or may be provided by existing structures in the borrowing 
language, or may even be innovated in the process of incorporation. 



It is not that the combinatoric relations surrounding a cultural item 
must also diffuse along with that item. Rather, it is that a cultural 
element cannot function or circulate if it does not have \textit{some} 
combinatoric relation to other cultural items in the same domain. This 
is the point that authors such as Sahlins and Eckert, mentioned above, 
have stressed for culture.



So, structuralist linguists like Donegan and Stampe are right when they 
say that a language \textquoteleft is not just a collection of autonomous parts', but 
this does not necessarily mean that a language is \textquoteleft a self-contained 
whole' (both quotes 1983, 1). Similarly, when cultural anthropologists 
refer to the \textquoteleft cultural totality' (Sahlins 1999), it is not yet clear 
what this really means. 



We never encounter whole systems except one fragment at a time. Our 
'partial experience' (Le Page and Tabouret-Keller 1985, 191) is not 
experience of the whole system, but nor is it experience of stand-alone 
items. Our direct experience of culture is of meaningful items in 
relations of functional incorporation and contextualization with other 
such items. 



Each such relation is, effectively, a combinatoric principle, like a 
norm for forming a grammatical sentence or for using a hammer and nail 
in the appropriate way. These relations are at the centre of the 
framework being proposed here. These relations are what is transmitted. 
Their inherent connection to some sort of system or field is entailed, 
but there is no pre-given size or outer limit of the relevant system or 
field.



Bloch (2000) says that old critiques of diffusionism in anthropology 
also serve as critiques of today's item-based accounts. I suggest that 
the problems are handled by the simple but crucial conceptual shift 
being proposed here. The relevant unit of cultural transmission (meme or 
whatever) is not \textquoteleft a piece'. The relevant unit is \textquoteleft a piece and its 
functional relation to a context'. This might seem obvious. But when we 
make it explicit, the fear of a disembodied view of cultural units go 
away. To be clear, the required conceptual move is not just to take 
items and put them in a context. Their relation to a context is part of 
what \textit{defines} them. 



\section{Solving the item/system problem in language}


Identifying the \textit{relation to context} as the common unit of 
analysis in items and systems is necessary but not yet sufficient. We 
need an account of how this scales up into large structured sets of such 
relations. Let us consider the question in connection to language. As a 
product of general mechanisms of social diffusion, each linguistic 
convention in a community has its own individual history. Each word, 
each morpheme, each construction has followed its own historical path to 
population-level conventionality. 



As Bloomfield (1933, 444) put it, \textquoteleft individual forms may have had very 
different adventures'. But as shown earlier, languages are not bundles 
of items. They are large, structured, systematic wholes, with 
psychological and intersubjective reality. Psychologically, languages 
exist as idiolects, cognitively represented and neurologically 
instantiated in individual speakers' bodies. Intersubjectively, 
languages exist at a community level to the extent that individuals' 
idiolects are effectively alike in structure and content, as 
demonstrated by the evidently tolerable degree of communicative success 
of normative practices of signifying and interpreting. We can now 
specify some forces that bring items together and structure them into 
systems.



\section{Centripetal and systematizing forces}


When we say that two people speak the same language, we mean that two 
individuals' knowledge of a language system is effectively (though never 
exactly) shared. This sharedness has come about because a large 
proportion of the same linguistic variants have been channelled, in a 
giant set, along the same historical pathways. The impression is that a 
language is passed down as a whole, transcending lifetime after lifetime 
of the individuals who learn and embody the system. 



This is the point made by Thomason \& Kaufman (1988): Normal social 
conditions enable children, as first language learners, to construct 
idiolects which effectively match the idiolects of the people they learn 
from-i.e., those who share the same household and immediate social 
environment, and, incidentally, who are most likely to share their 
genes. Normal transmission is what allows historical linguists to 
abstract from the fact that each linguistic variant has its own career, 
instead treating the whole language as having a single 
spatial-historical trajectory. 



In many cases this is a reasonable-indeed successful-methodological 
presumption (cf. Haspelmath 2004). But work such as Le Page \& 
Tabouret-Keller (1985) and Thomason \& Kaufman (1988) shows that in 
situations other than those of normal transmission, linguistic items do 
not always travel together, but may follow separate paths, making 
visible what is always true but usually obscured by items' common 
destiny in practice, namely: Each item has its own history. 



Genealogical continuity in language change is typically taken to be the 
norm, and whenever we see that linguistic systems are relatively 
permeable, for instance in certain language contact situations where the 
components of languages are prised apart, special explanations are 
demanded. 

\textit{On normal transmission:}

The idea that ‘a language’ is ‘inherited’ by a child from her parents is an implausible representation of what actually goes on in language acquisition. The idiolect of the child is not donated in an ovular/seminal bundle like DNA. Patterns of clausal subordination do not unfold in the child like the shape of his nose. Ideas for ways of saying things must be learned, painstakingly, constructed, and maintained through constant practice. The passage of this ‘transmission’ is through air, over days, weeks, months, years, with great interference and noise. Every bit of the idiolect’s structure has to be passed over and constructed from scratch by the recipient. This task is made possible by the sheer deluge of linguistic data which people are endlessly exposed to, and endlessly participating in, Hayakawa’s ‘Niagara of words’ (1978:12ff). That everyone in every community is so constantly hammering each other with words and expressions and sentences and ways of saying things keeps levels of exposure so high that a great degree of convergence in people’s linguistic habits is maintained. And since we use so many signs from day to day, this mutual deluge of speech is required in order to keep up sufficient exposure and keep everyone in the network familiar with all the words and turns of phrase. With this requirement for a mutual intra-network deluge of forms through constant daily contact, we only have enough time in the day to maintain interaction with a given number of people. Dunbar (1996) has hypothesised that pre-linguistic human ancestors created language itself as a way to lessen time pressure due to the need to groom an ever greater number of social associates. Sustaining a social network by means of linguistic contact remains akin to grooming, and remains time-consuming, particularly when the deluge of forms has to be maintained. Where personal ‘exchange’ or ‘strong’ ties  are involved, we remain necessarily oriented towards a limited group, and the size of networks is constrained by the time it takes to maintain these exchange tie relationships. However, the number of non-personal exposure ties (seeing and hearing, especially due to television and high population density) is potentially massive. Writing, and especially television more recently, have drastically changed the proportion of personal exposure sources and non-personal exposure sources. Television now means that people around the globe share a large amount of non-personal exposure ties (i.e. they are exposed to the same semiotic structures).

Thomason and Kaufman (1988) invoke a notion of ‘normal transmission’, defined ‘by exclusion’ (1988:10), with reference to level of ‘perfection’ of reproduction of all systems of a language in the children’s idiolects. In ‘normal transmission’, linguistic categories of ‘other people’ do not enter into (or only minimally enter into) the long-term project of exposing children to, and getting children to replicate in their performance, the full range of categories in ‘a language’, in a process of mediated construction of an idiolect highly convergent with the parents’ idiolects. 
‘Normal transmission’ in Thomason and Kaufman’s sense is assumed to be a ‘social fact’ (1988:12), but is defined by formal facts about language acquisition by children in a given community:

[A] claim of genetic relationship [between a ‘parent’ and a ‘daughter’ language] entails systematic correspondences in all parts of the language because that is what results from normal transmission: what is transmitted is an entire language---that is, a complex set of interrelated lexical, phonological, morphosyntactic, and semantic structures. (Thomason and Kaufman 1988:11)

	I gather that for Thomason and Kaufman, to say that a ‘genetic’ relationship holds between ‘parent’ and ‘daughter’ languages is to use a metaphor, and that to use this metaphor is considered harmless only in case such a degree of the parents’ idiolects is reproduced in the idiolect of the child that it is \textit{as if} the latter were a replica, born according to the plan of the former (i.e. in the case of ‘normal transmission’). But Thomason and Kaufman continue to rely on this metaphor themselves when they refer to ‘transmission of an entire language’. Thomason and Kaufman’s ‘normal transmission’ entails a relentless and focussed linguistic sign deluge from people of whom the learner is one.



But we must also ask: How to explain the relative \textit{im}
permeability of linguistic systems in \textit{normal} circumstances? 
\textit{Stability }in conventional systems is no less in need of 
explanation than variation or change (Bourdieu 1977; Sperber 1996; 
Sperber and Hirschfeld 2004)\textit{. }What are the forces which 
cause linguistic variants to follow en masse a single path of diffusion 
and circulation, and to cohere as structured systems? Let us briefly 
consider three such forces.



\textit{Sociometric closure}



A first centripetal force is \textit{sociometric closure, }arising 
from a trade-off between strength and number of relationship ties in a 
social network. If a person is to maintain a social relationship, she 
has to commit a certain amount of time to this, and since time is a 
finite resource, this puts a structural constraint on the possible 
number of such relationships one can maintain (Hill and Dunbar 2003). 
The result is a relatively closed circulation of currency within a 
social economy of linguistic items, causing individuals' inventories of 
items (i.e., their vocabularies, etc.) to overlap significantly, or to 
be effectively identical, within social networks.



This helps to account for how individuals in regular social association 
can have a common set of variants, but it does not account for the 
tightly structured nature of the sets of relations among those items. We 
turn now to two reflexive forces of relational systematization inherent 
to grammar, in both the paradigmatic (�5.1.2) and syntagmatic (�5.1.3) 
dimensions.



\textit{Trade-off effects}



One systematizing force comes from functional trade-off effects that 
arise when a goal-oriented person has ongoing access to a set of 
alternative means to similar ends. When different items come to be used 
in a single functional domain, those items can become formally and 
structurally affected by their relative status in the set. This happens 
because the items compete for a single resource-namely, our selection of 
them as means for our communicative ends. 



When Zipf (1949, 19ff) undertook \textquoteleft a study of human speech as a set of 
tools', he compared the words of a language with the tools in an 
artisan's workshop. Different items have different functions, and thus 
different relative functional loads. In a vocabulary, Zipf (1949) 
argued, there is an \textquoteleft internal economy' of words (p21), with trade-offs 
that result in system effects such as the observed correlation between 
the length of a word (relative to other words) and the frequency of use 
of the word (relative to that of other words). 



Zipf reasoned that \textquoteleft the more frequent tools will tend to be the lighter, 
smaller, older, more versatile tools, and also the tools that are more 
thoroughly integrated with the action of other tools' (Zipf 1949:73). He 
showed that the more we regard a set of available means as alternatives 
to each other in a functional domain, the more they become defined in 
terms of each other, acquiring new characteristics as a result of their 
role in the economy they operate in. The upshot is this: The more we 
treat a set of items as a system, the more it becomes a system.



\textit{Item-utterance fit, or content-frame fit}



Another key conduit and filter for grammatical structure is grammatical 
structure itself. The \textit{utterance }is a core structural locus in 
language, providing narrow contextualization for the interpretation of 
linguistic variants, and serving as an essential ratchet between item 
and system. As Kirby writes, although \textquoteleft semantic information' is what 
linguistic utterances most obviously convey, \textquoteleft there is another kind of 
information that can be conveyed by any linguistic production, and that 
is information about the linguistic system itself' (Kirby 2013, 123). 



He adds: \textquoteleft When I produce the sentence "\,these berries are good" I may 
be propagating cultural information about the edibility of items in the 
environment via the content of the sentence. At the same time I may also 
be propagating information about the construction of sentences in my 
language' (Kirby 2013, 123). In this way, the utterance is a basic frame 
for replication of linguistic variants (Croft 2000). 



Item-utterance fit is the structural fit between diffusible types of 
linguistic item and the token utterances in which they appear. It is an 
instance of the more general \textquoteleft content-frame' schema (Levelt 1989) also 
observed in phonology (MacNeilage 1998; see Enfield 2013, p54-55), and a 
case of the \textquoteleft functional relation to context' defined above as the core 
common property of items and systems. Here we see it is not just a 
common property, but it is the very property that \textit{links }them. 
In this way, an utterance provides an incorporating and contextualizing 
frame for the diffusion of replicable linguistic variants, \textit{and
} a frame for the diffusion of the combinatoric rules from which the 
higher-level system is built. 



\section{A solution to the item/system problem?}


The above considerations suggest that the item/system problem can be 
solved with reference to three forces that apply in the context of the 
biased transmission of cultural items: (1) bundling of items arising 
from the population-level effects of sociometric biases, (2) 
system-forming effects arising from the treatment of elements in a set 
as alternative means to related functional ends, and (3) transmission of 
the combinatoric properties of items via context biases and the relation 
of item-utterance fit. We can expect there to be analogous relations to 
item-utterance fit in the domain of culture (think, for instance, of 
systems of social relations in kinship, or systems of material culture 
and technology in households and villages). 



Zipf's (1949) analogy is useful here. For his \textquoteleft economy of tools-for-jobs 
and jobs-for-tools' to get off the ground, one first needs a \textit{
workshop}, somewhere the set of tools can be assembled in one place 
and made accessible to an agent with a set of goals. In language and 
culture, this is achieved by sociometric closure (see 5.1.1, above). 
Then one begins to work with the set of tools, using them as alternative 
means to similar ends (section 5.1.2 above). These tools will, whether 
by design or by nature, enter into relations of incorporation and 
contextualization that define their functional potential (section 5.1.3 
above). 



In short, once we get an \textit{inventory} of items that are 
functional within a given domain, they will naturally enter into the 
\textit{paradigmatic} and the \textit{syntagmatic} relations that 
define semiotic systems in the classical sense.









\newpage


\newpage